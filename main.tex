%%%%%%%%%%%%%%%%%%%%%%%%%%%%%%%%%%%%%%%%%
%  My documentation report
%  Objetive: Explain what I did and how, so someone can continue with the investigation
%
% Important note:
% Chapter heading images should have a 2:1 width:height ratio,
% e.g. 920px width and 460px height.
%
%%%%%%%%%%%%%%%%%%%%%%%%%%%%%%%%%%%%%%%%%

%----------------------------------------------------------------------------------------
%	PACKAGES AND OTHER DOCUMENT CONFIGURATIONS
%----------------------------------------------------------------------------------------

\documentclass[11pt,fleqn]{book} % Default font size and left-justified equations

\usepackage[top=3cm,bottom=3cm,left=3.2cm,right=3.2cm,headsep=10pt,letterpaper]{geometry} % Page margins
\usepackage[dvipsnames]{xcolor}
\usepackage{lipsum} % Required for specifying colors by name

% Font Settings
\usepackage{avant} % Use the Avantgarde font for headings
%\usepackage{times} % Use the Times font for headings
\usepackage{mathptmx} % Use the Adobe Times Roman as the default text font together with math symbols from the Symbol, Chancery and Computer Modern fonts

\usepackage{microtype} % Slightly tweak font spacing for aesthetics
\usepackage[utf8]{inputenc} % Required for including letters with accents
\usepackage[T1]{fontenc} % Use 8-bit encoding that has 256 glyphs

\usepackage[french]{varioref}
\usepackage[french]{babel}

\usepackage{wrapfig}
\usepackage{listings}
\usepackage{multicol}
\usepackage{multirow}
\usepackage{colortbl}
\usepackage{booktabs}
\newcommand{\tabitem}{~~\llap{\textbullet}~~}
\usepackage{hyperref}
\usepackage{qrcode}
\usepackage{soul}
\usepackage{tabto}
\usepackage{multienum}


\definecolor{deepblue}{rgb}{0.0,0.0,0.5}
\definecolor{deepred}{rgb}{0.6,0,0}
\definecolor{deepgreen}{rgb}{0,0.5,0}
\definecolor{ocre}{RGB}{51,102,0} 
\definecolor{lightgray}{RGB}{229,229,229} 
\definecolor{palerod}{RGB}{238,232,170}
\definecolor{verttelecom}{RGB}{171,180,0}

\newcommand\pythonstyle{\lstset{
language=Python,
basicstyle=\ttfamily\footnotesize,
morekeywords={self},              % Add keywords here
frame=tb,                         % Any extra options here
showstringspaces=false
}}

\definecolor{backcolour}{rgb}{0.95,0.95,0.92}

\newcommand\termctyle{\lstset{
frame=tb,                         % Any extra options here
showstringspaces=false
}}


% Python environment
\lstnewenvironment{python}[1][]
{
\pythonstyle
\lstset{#1}
}
{}

\lstnewenvironment{termc}[1][]
{
\lstset{#1}
}
{}


% RFC
\newcommand\rfc[1]{\href{http://www.ietf.org/rfc/rfc#1.txt}{\textcolor{blue}{RFC #1}\index{RFC #1}}}
\newcommand\pfunction[2]{\texttt{#2}\index{Module Python!#1!#2}}

% boot.py

\newcommand\glos[1]{\gls{#1}\index{#1}}
\newcommand\pprog[2]{\href{https://github.com/ltn22/PLIDObis/blob/master/#2/#1}{\texttt{#1}}\index{Programmes Python!#1}}
\newcommand\lprog[2]{\href{https://github.com/ltn22/PLIDObis/blob/master/#2/#1}{\texttt{#1}}\index{Programmes micro-python!#1}}

% QUESTION

\usepackage[most]{tcolorbox}
\usepackage{ifthen}

\provideboolean{Response}\setboolean{Response}{true}

\newcommand{\Correct}[1]{\ifthenelse{\boolean{Response}}{#1}{\textbf{#1}}}
\newcommand{\Wrong}[1]{\ifthenelse{\boolean{Response}}{#1}{\textcolor{black!20}{#1}}}

\newwrite\tempfile
\immediate\openout\tempfile=questions.tex


\newtcbtheorem[auto counter,number within=section]{theo}%
  {Question}{fonttitle=\bfseries\upshape, 
     arc=0mm, colback=blue!5!white,colframe=blue!75!black}{Question}
     
\newcommand\Question[3]{
\begin{theo}{#1}{summation}
#2
\immediate\write\tempfile{\noexpand\textbf{Question \thetcbcounter {} page \thepage} {} }
\immediate\write\tempfile{\unexpanded{#2}\noexpand\vspace{1em}\noexpand\newline}
\immediate\write\tempfile{\unexpanded{#3}\noexpand\newline\noexpand\newline}

\end{theo}
}

% MATHS PACKAGE
\usepackage{amsmath,tikz}
\usetikzlibrary{matrix}
\newcommand*{\horzbar}{\rule[0.05ex]{2.5ex}{0.5pt}}
\usepackage{calc}

% VERBATIM PACKAGE
\usepackage{verbatim}

\usepackage{tikz}

\usetikzlibrary{automata}
\usetikzlibrary[shadows]
\usetikzlibrary{shapes}
\usetikzlibrary[decorations.footprints] 
\usetikzlibrary{decorations.pathmorphing}
\usetikzlibrary{decorations.pathreplacing}
\usetikzlibrary{decorations.text}
\usetikzlibrary {arrows}
\usetikzlibrary{patterns}
\usetikzlibrary{calc}
\usetikzlibrary{external}

\usepackage{tikz-timing}

% Acronyms

\usepackage{makeidx}
\makeindex

\usepackage{acronym}

\let\oldac\ac
\renewcommand*{\ac}[1]{\oldac{#1}\index{#1}}

\newcommand\Index[1]{\textbf{#1}\index{#1}}

% Bibliography
\usepackage[style=alphabetic,sorting=nyt,sortcites=true,autopunct=true,babel=hyphen,hyperref=true,abbreviate=false,backref=true,backend=biber]{biblatex}
\addbibresource{bibliography.bib} % BibTeX bibliography file
\defbibheading{bibempty}{}

\input{structure} % Insert the commands.tex file which contains the majority of the structure behind the template



\newcommand\pythonlst[2][]{
\lstinputlisting[language=Python, backgroundcolor=\color{palerod},   basicstyle=\footnotesize\ttfamily,
  keywordstyle=\bfseries\color{green!40!black},
  commentstyle=\itshape\color{purple!40!black},
  identifierstyle=\color{blue},
  stringstyle=\color{orange}, caption=#2,
  numbers=left, numberstyle=\tiny, stepnumber=2, numbersep=5pt, frame=single, #1] {Programs/#2}\index{Programmes Python!#2}
  }

\newcommand\pythonnxt[2][]{
\lstinputlisting[language=Python, backgroundcolor=\color{palerod},   basicstyle=\footnotesize\ttfamily,
  keywordstyle=\bfseries\color{green!40!black},
  commentstyle=\itshape\color{purple!40!black},
  identifierstyle=\color{blue},
  stringstyle=\color{orange},
  numbers=left, numberstyle=\tiny, stepnumber=2, numbersep=5pt, frame=single, #1] {Programs/#2}
  }
  
  
\newcommand\pycomlst[2][]{
\lstinputlisting[language=Python, backgroundcolor=\color{gray!10},   basicstyle=\footnotesize\ttfamily,
  keywordstyle=\bfseries\color{green!40!black},
  commentstyle=\itshape\color{purple!40!black},
  identifierstyle=\color{blue},
  stringstyle=\color{orange}, caption=#2,
  numbers=left, numberstyle=\tiny, stepnumber=2, numbersep=5pt, frame=single, #1] {Programs/#2}\index{Programmes micro-python!#2}
  }

\newcommand\pycomnxt[2][]{
\lstinputlisting[language=Python, backgroundcolor=\color{gray!10},   basicstyle=\footnotesize\ttfamily,
  keywordstyle=\bfseries\color{green!40!black},
  commentstyle=\itshape\color{purple!40!black},
  identifierstyle=\color{blue},
  stringstyle=\color{orange},
  numbers=left, numberstyle=\tiny, stepnumber=2, numbersep=5pt, frame=single, #1] {Programs/#2}
  }



\newcommand\Youtube[1]{\begin{tcolorbox}[colback=red!5,colframe=red!75!black,title=Youtube, width=3cm]\href{#1}{\qrcode{#1}}\end{tcolorbox}}

\newcommand\fulluri[2]{\href{#2}{#1}\footnote{\url{#2}}}
%%%%%%%

\provideboolean{allchap}\setboolean{allchap}{false}

\newcommand\Input[1]{\ifthenelse{\boolean{allchap}}{\input{#1}}{}}


%-----------

\provideboolean{lfrench}\setboolean{lfrench}{false}
\provideboolean{lenglish}\setboolean{lenglish}{true}

\newcommand\lgf[1]{\ifthenelse{\boolean{lfrench}}{#1}{}}
\newcommand\lge[1]{\ifthenelse{\boolean{lenglish}}{#1}{}}

\begin{document}

\let\cleardoublepage\clearpage

%----------------------------------------------------------------------------------------
%	TITLE PAGE
%----------------------------------------------------------------------------------------

\begingroup
\thispagestyle{empty}
\AddToShipoutPicture*{\put(0,0){\includegraphics[scale=1.25]{v}}} % Image background
\centering
\vspace*{5cm}
\par\normalfont\fontsize{35}{35}\sffamily\selectfont
\textbf{PROGRAMMER L'INTERNET DES OBJETS }\\
{\LARGE }\par % Book title
\vspace*{1cm}
{\Huge Laurent TOUTAIN}\par % Author name
\endgroup

%----------------------------------------------------------------------------------------
%	COPYRIGHT PAGE
%----------------------------------------------------------------------------------------

\newpage
~\vfill
\thispagestyle{empty}

%\noindent Copyright \copyright\ 2014 Andrea Hidalgo\\ % Copyright notice

\noindent \textsc{IMT Atlantique}\\

\noindent Basé sur le MOOC PLIDO.\\ % License information

\noindent \textit{Publié le \today} % Printing/edition date

%----------------------------------------------------------------------------------------
%	TABLE OF CONTENTS
%----------------------------------------------------------------------------------------


\chapterimage{pano-5.jpg} % heading image

\pagestyle{empty} % No headers

\renewcommand\contentsname{Table des Matières}
\renewcommand{\bibname}{Bibliographie}

\cleardoublepage
\tableofcontents% Print the table of contents itself

%\cleardoublepage % Forces the first chapter to start on an odd page so it's on the right

\pagestyle{fancy} % Print headers again

%----------------------------------------------------------------------------------------
%	CHAPTERS
%----------------------------------------------------------------------------------------
\cleardoublepage

\chapter*{Acronymes}
\begin{multicols}{2}
\begin{acronym}
\acro{3GPP}{3rd Generation Partnership Project}
\acro{ADSL}{Asymmetric Digital Subscriber Line}
\acro{CoAP}{Constrained Application Protocol}
\acro{CRC}{Cyclic Redundancy Check}
\acro{HTML}{HyperText Markup Language}
\acro{HTTP}{HyperText Transport Protocol}
\acro{HTTPS}{HyperText Transport Protocol Secure}
\acro{IBAN}{International Bank Account Num
ber}
\acro{IEEE}{Institute of Electrical and Electronics Engineers}
\acro{IETF}{Internet Engineering Task Force}
\acro{IoT}{Internet of Things}
\acro{IP}{Internet Protocol}
\acro{IPv4}{Internet Protocol version 4}
\acro{IPv6}{Internet Protocol version 6}
\acro{IRI}{International Resource Identifier}
\acro{ISBN}{International Standard Book Number}
\acro{ISO}{International Standardization Organization}
\acro{JSON}{JavaScript Object Notatin}
\acro{LCIM}{Levels of Conceptual Interoperability Model}
\acro{LNS}{LoRaWAN Network Server}
\acro{MQTT}{Message Queuing Telemetry Transport}
\acro{NAT}{Network Address Translation}
\acro{REST}{REpresentational State Transfer}
\acro{RFC}{Request For Comments}
\acro{RNIPP}{Répertoire National d'Identification des Personnes Physiques}
\acro{SCEF}{Service Capability Exposure Function}
\acro{STIC}{Sciences et Technologies de l’Information et de la Communication}
\acro{TCP}{Transmission Control Protocol}
\acro{TNT}{Télévision Numérique Terrestre}
\acro{UDP}{User Datagram Protocol}
\acro{URI}{Universal Resource Identifier}
\acro{URL}{Univeral Resource Locator}
\acro{URN}{Univeral Resource Name}
\acro{W3C}{World Wide Web Consortium}
\acro{WWW}{World Wide Web}
\acro{XML}{Extensible Markup Language}
\end{acronym}
\end{multicols}

\setboolean{allchap}{true} % true: take all, false take nothing only /input

\setboolean{lfrench}{false}
\setboolean{lenglish}{true}

\Input{Part00-liminaire}
\chapterimage{pano-tv1.png} % Chapter heading image
\cleardoublepage
\chapter{LES BASES DE L'INTERNET DES OBJETS (IOT)}

\section{Introduction}

  \vspace{1em}
 \begin{wrapfigure}{r}{3cm}
\Youtube{https://youtu.be/9edD2jEF3vM}
\end{wrapfigure}
 Dans cette première partie du cours, nous allons poser les bases de ce qu'est l'internet des objets (\ac{IoT} in english). Qu'est-ce qu'on entend par \ac{IoT} dans le cadre de livre~? Quelles sont les problématiques auxquelles doit répondre l'\ac{IoT} et son évolution aujourd'hui~? Quelles sont les technologies, les architectures, les protocoles sous-jacents qui seront utilisés dans cet ouvrage~? 
 
 Pour cela, nous allons faire un parallèle entre la manière dont l'internet a intégré la télévision et ce que l'on vit actuellement avec l'internet des objets. 
 
 \subsection{Réseaux dédiés}
   \vspace{1em}

\begin{wrapfigure}{r}{6cm}
\centerline{\includegraphics[width=.4\columnwidth]{Pictures/illu-propa.png}}
\end{wrapfigure}

 Dans les années 50, la télévision est devenue très populaire et presque tous les foyers ont acheté un téléviseur pour regarder leurs programmes préférés. Des réseaux de transmission dédiés ont été déployés partout sur la planète. En fait chaque type de communication avait son propre réseau un pour la radio un pour le téléphone un pour le télex,...
 


Dans les années 80, l'internet a vu le jour, mais les vitesses de transmission était faible et le réseau était limitée aux chargements de fichiers. Dans les années 90, des images et l'hypermédia avec le \ac{WWW} sont apparus, en lien avec l'augmentation des débits. Toujours dans les années 90, l'augmentation en puissance des microprocesseurs a permis la numérisation du signal télé. Les téléviseurs ont commencé à inclure des microprocesseurs et les réseaux sont passés d'une transmission analogique au numérique ; mais ils sont restés dédié à cet usage unique diffuser la télévision. Avec l'entrée dans le nouveau millénaire, l'internet a gagné en débit avec l'\ac{ADSL} et des fibres optiques. Il était possible d'intégrer des images dans les pages web mais la qualité était médiocre. Au même moment des centaines de canaux de télévision diffusaient en haute résolution leur programme via satellite ou par \ac{TNT}. De nos jours les communications internet ont gagné en vitesse et en qualité et certains pays ont coupé la \ac{TNT} et choisi de ne transmettre leurs programmes que par internet. En fait l'utilisation d'internet n'est pas uniquement un changement de réseau de distribution c'est aussi un changement majeur dans les usages et les applications. Vous pouvez regarder la télévision sur votre téléphone portable ou même regarder votre série favorite quand vous le voulez, à la demande. 

  \vspace{2em}

\subsection{3 phases technologiques}
   \vspace{1em}

\begin{wrapfigure}{r}{6cm}
\centerline{\includegraphics[width=.4\columnwidth]{Pictures/illu-verticals.png}}
\end{wrapfigure}
De cet exemple, on peut définir trois phases dans le développement d'une technologie. Dans la première phase, un réseau spécifique est construit pour un usage bien défini. On appelle cela une approche \Index{verticale} ; une technologie est dédiée à un seul usage. Il est difficile d'échanger de l'information entre deux verticales. On fait aussi référence à des \Index{silos} car ils sont isolés. Dans une seconde phase, les verticales commencent à intégrer des technologies communes mais pas d'une manière coordonnée. Elles ne peuvent toujours pas communiquer facilement car elles n'ont pas fait les mêmes choix.

~

\begin{wrapfigure}{r}{6cm}
\centerline{\includegraphics[width=.4\columnwidth]{Pictures/ill-horizontales.png}}
\end{wrapfigure}
Dans une dernière phase, des verticales se coordonnent pour converger vers les mêmes technologies en définissant des règles et des usages communs. Ceci dans le but de réduire les coûts ou d'augmenter leur impact dans ce cas. On parle d'horizontal\index{Horizontale} car elle couvre plusieurs secteurs. L'internet est devenu une de ces horizontales pour beaucoup de services. L'internet des objets suit ce même mouvement. Des solutions particulières ont émergé pour résoudre des besoins spécifiques en agriculture, dans l'automobile, dans la santé, dans l'énergie. Quand les réseaux internet ont permis des communications à faible coût, l'architecture de l'internet a été prise en compte mais sans compatibilité. Le changement qu'on vit actuellement est la définition de fonctionnalités communes à différents domaines. Le but étant de réduire les coûts mais également de croiser les informations pour une meilleure gestion du processus industriel et un meilleur usage des ressources
 

  \vspace{2em}

\section{L'Internet des Objets}

  \vspace{1em}
  
 Comment définir l’internet des objets~? Ou plutôt, quel internet des objets allons-nous étudier~? L’ambiguïté des deux termes "internet" et "objets" impose une définition plus précise ; ou du moins une classification pour mieux comprendre à quoi l’on fait référence.

  \vspace{1em}


L’internet est maintenant totalement intégré dans nos vies, pour le travail, l’enseignement, pour les distractions. On l’utilise à la maison ou au travail sur nos ordinateurs, et on l'emporte avec nous de plus en plus avec nos smartphones. 

Chacun a sa définition de ce qu’est internet. Pour le grand public, il peut s’agir d’applications très populaires comme Facebook, Tik-Tok, Netflix, Zoom. Pour certains, un peu plus technophiles, l’internet peut être confondu avec le Web auquel on accède via Chrome ou Firefox. Les techniciens parleront de protocoles comme IP, TCP, HTTP, et d’adresses comme les adresses IP ou les URL.

~~

Comme ce livre est orienté technologie, notre approche relève plutôt de cette dernière catégorie. Nous verrons comment des protocoles développés il y a une vingtaine d’années pour des ordinateurs peuvent s’appliquer à d’autres dispositifs qu’il nous reste à définir.

L'objectif de l'internet des objets est de poursuivre l'intégration du réseau Internet pour permettre à autre chose que des ordinateurs d'échanger des données. Le but principal est d'optimiser les processus pour qu'ils soient plus efficaces pour économiser des ressources ou d'augmenter la productivité. Il s'agit donc d'un internet enfoui, loin du frigo ou de la montre connectés, qui vont remonter des informations avec une infrastructure ou d'autres équipements. On peut imaginer des capteurs dans une usine pour contrôler la production, des voitures connectées qui vont dialoguer pour éviter les collisions, la mesure du taux de remplissage des bennes de recyclage dans une ville pour optimiser les circuits de collecte, la surveillance du degré d'humidité d'un champ pour réduire la consommation d'eau...


L’internet des objets peut se résumer de la manière suivante : utiliser des protocoles développés pour des ordinateurs et maintenant des téléphones portables (plus puissants que les ordinateurs utilisés par l’internet à ses débuts) mais dans des environnements plus contraints. En effet, les lois de Moore, définissant les puissances de traitement des processeurs ainsi que la diminution continue des coûts de la mémoire, nous ont permis de doter les petits objets de ressources comparables à celles des ordinateurs d’il y a trente ans.

L’internet des objets, c’est résoudre l’équation suivante : continuer à faire la même chose car tous les systèmes d’information actuels utilisent les mêmes principes mais le faire différemment car ces principes sont trop coûteux en énergie, en temps de calcul et échange de données.

L’\ac{IoT}, l’internet des objets ou des choses, est une architecture globale permettant à des objets (équipements de même type ou non), d’interagir de manière autonome via internet. Cette interaction :

\begin{itemize}
    \item est réalisée, par construction, au travers d’un réseau internet, ce qui implique généralement que les objets/choses soient pourvus d’une adresse IP ;
    \item est relatif à des commandes (opérations de contrôles ou appels de fonctions) ou des échanges de données ou d’informations.
\end{itemize}

  \vspace{1em}


Ce nouveau paradigme \ac{STIC} qu’est l’IoT est une convergence de nombreux domaines d’applications tels que : les maisons ou bâtiments intelligents, les villes du futur, l’industrie du futur (industrie 4.0), l’énergie, les systèmes de transport, l’agriculture, la eSanté, etc., vers une suite protocolaire réduite, interopérable et sécurisée. Le mouvement est en marche et, vu le nombre d'acteurs concernés, va prendre plusieurs années. Mais les bases sont déjà bien établies et c'est ce que vous allez apprendre dans cet ouvrage.

 \vspace{2em}
 
\section{Le problème}

  \vspace{1em}

Un des problèmes que rencontre l’internet des objets, c’est que l’IoT ne démarre pas ex-nihilo. Maintenant que les technologies qui ont fait le succès de l'Internet sont matures, il ne s'agit pas juste de les appliquer à un nouveau domaine. Des objets étaient capables de communiquer bien avant qu’internet n’existe. Chaque secteur a déjà développé ses solutions, plus ou moins standards, plus ou moins propriétaires.

La figure~\vref{fig-bazar} reprend un certain nombre de travaux et de groupes qui spécifient les protocoles pour l’internet des objets. 

\begin{figure}[tbp]
\centerline{\includegraphics[width=1\columnwidth]{Pictures/IOTBazar_Domaines.jpeg}}
\caption{Quelques standards de l'IoT}
\label{fig-bazar}
\end{figure}

Sans entrer dans les détails, on voit que certains logos se retrouvent à plusieurs emplacements, qu’il y a pour chaque secteur une profusion de solutions qui nuisent à l’interopérabilité et aux évolutions. L’internet des objets, dans son acception la plus large, consiste à simplifier cette architecture, comme l’internet l'a fait il y a quelques années dans le domaine des télécoms en simplifiant ce modèle et en permettant à ces différents acteurs de converger vers une architecture commune et un ensemble de solutions plus réduit.

Cela ne veut pas nécessairement dire moins d’acteurs, mais une plus grande cohérence dans les choix technologiques.

La figure~\vref{fig-bazar-OS} analyse l’IoT par domaine d’application en se focalisant sur la composante réseau. L’IoT et les objets connectés sont des systèmes complexes pour lesquels les solutions open source, les alliances entre industriels, les organismes de standardisation sont toujours fragmentés ; cependant de manière moins importante, montrant que la structuration est en marche.

\begin{figure}[tbp]
\centerline{\includegraphics[width=1\columnwidth]{Pictures/IOTBazar_OpenSources.jpeg}}
\caption{Quelques applications Open Source pour l'IoT}
\label{fig-bazar-OS}
\end{figure}

Cette fragmentation de l’écosystème est paradoxale. Si l’on résume sommairement, les solutions proposées reviennent à interroger un équipement sur le terrain pour accéder à une valeur, la traiter et renvoyer une commande pour interagir avec l’environnement.

Pourquoi cette foison de solutions différentes~? Cela peut venir des besoins de fiabilité, de sécurité, de portée de la donnée, mais cela vient aussi de l’histoire. La communication avec des objets est tout aussi ancienne que la communication entre ordinateurs (qui sont eux-mêmes des objets). Mais à l’époque, chaque domaine a suivi sa propre voie en spécialisant les solutions pour répondre à ses besoins propres. Il en résulte des solutions optimisées pour un domaine particulier. Mais chaque fois qu’il faut modifier une technologie, le travail doit être adapté pour chaque domaine, introduisant des coûts et des délais supplémentaires.

De même, chaque domaine ayant sa propre représentation des données, il est relativement difficile de les combiner pour avoir une vision plus globale. On arrive donc à des systèmes fermés, chers, peu évolutifs, mais optimisés pour les tâches qu’ils ont à réaliser.

Au fil du temps, les protocoles de l’internet ont pu être utilisés, mais il s’agit surtout de complémenter les technologies existantes sans qu’il soit possible d’interconnecter deux domaines.


  \vspace{2em}

\section{Evolution de l'IoT}

  \vspace{1em}

L’exemple de l’évolution du réseau de télévision est éloquent. À ses débuts, ce réseau est analogique et hautement spécialisé pour diffuser les programmes transportés sur des signaux analogiques sur des équipements spécialisés, les téléviseurs.

Avec les progrès des processeurs informatiques, il devient possible de transporter les données en utilisant un codage numérique. Mais des réseaux spécialisés restent nécessaires, l’Internet n’offrant pas la même qualité.



La dernière étape consiste à intégrer ces flux dans l’internet classique. Cela devient possible par la montée en débit des réseaux filaires et radio (Wi-Fi, 4G...). 

Cette mutualisation des accès via l’internet permet non seulement une réduction des coûts, mais aussi l'apparition de nouveaux usages comme la télévision sur les téléphones portables ou les séries à la demande.

L’internet des objets suit la même voie. En plus des technologies spécifiques, les protocoles de l’internet sont intégrés, mais en les adaptant aux contextes du secteur.  Nous sommes actuellement en train de vivre la convergence vers un ensemble réduit de protocoles, une standardisation de la représentation de la donnée, et son traitement sur des plates-formes plus génériques. 

Le déclencheur n’est pas la montée en débit comme pour la télévision, mais la possibilité d'avoir des équipements peu chers, aux capacités réduites par rapport à l'informatique traditionnelle et autonomes énergétiquement, tout en ayant une meilleure intégration dans les systèmes d'information actuels.  

  \vspace{2em}

\section{Des objets contraints}

  \vspace{1em}

Avec les progrès de l’électronique, les processeurs deviennent de plus en plus puissants et les super-ordinateurs d’hier sont maintenant dans une montre ou un smart-phone. 

Pour l’internet des objets, la logique est un peu différente. La loi de Moore va induire une réduction des coûts de fabrication plutôt qu'une augmentation des puissances de traitement. Le principal critère pour un internet des objets massif reste l’énergie ; connecter un appareil à une source d’énergie ou recharger une batterie a un coût. Augmenter la vitesse du processeur ou la taille de la mémoire induit une plus grande consommation d’énergie de l'objet. On peut donc s’attendre à une certaine stabilité des performances des objets car ceux-ci resteront limités en performances.

Les objets sont généralement limités en termes de puissance de traitement, de mémoire et d’énergie. Selon le standard de l’IETF \rfc{7228}, les dispositifs peuvent être répartis en trois classes qui se retrouvent aussi dans la segmentation des processeurs :

\begin{itemize}
\item La classe 0, avec moins de 10 ko de mémoire volatile pour stocker les données temporaires et 100 ko de mémoire Flash pour stocker le code informatique de l'objet. C'est l’équivalent d’un Arduino UNO (2 ko de RAM et 32 ko de Flash). Il est presque impossible d’installer à la fois les protocoles utilisés pour communiquer sur Internet (même de manière restreinte) et les applications qui tournent dessus. 
\item La classe 1 a environ 10 Ko de RAM et 100 Ko de Flash. Avec une adaptation, il est possible d’y installer une pile IP. Il s'agit par exemple d'équipement comme le Pycom Lopy4 que nous utiliserons par la suite (et qui se situe dans la limite haute) sur lequel le système d'exploitation est minimal. Ainsi, le Pycom utilise une version simplifiée du langage Python (micro-python) qui permet de l'adapter à la limitation du système.
\item La classe 2 est moins restreinte avec au moins 50 ko de RAM et 250 ko de Flash (comme un Raspberry Pi). Le système d’exploitation Linux peut fonctionner sur ces appareils. Par conséquent, il y a peu de limitations sur la pile IP et les applications s’y exécutant.
Les appareils de classe 1 ont trop de restrictions pour utiliser les protocoles définis pour des objets plus gros. L’\ac{IETF}, l'organisme qui standardise les protocoles de l'internet, a proposé une révision de sa pile de protocoles afin d’adapter sa pile protocolaire à un environnement contraint.
\end{itemize}


La figure~\vref{fig-encap} résume les moyens d'interconnexion suivant la classe de l'objet~:
Un appareil de classe 0 ne peut pas utiliser directement l’internet pour échanger des informations, d’où la nécessité d’installer une passerelle pour capter le trafic et l’envoyer sur l’internet. Il ne possède pas directement d'adresse IP. Les passerelles LoRaWAN \ac{LNS} et \acs{3GPP} \ac{SCEF} agissent dans ce sens (nous reviendrons là-dessus). Les données produites sont encapsulées par ces passerelles dans des protocoles comme \ac{HTTP} ou \ac{MQTT}, que nous verrons également dans la suite du cours.

Les appareils de classe 1 peuvent aussi utiliser une passerelle pour s’interconnecter à l’internet traditionnel, mais plutôt que d'encapsuler les données produites dans d'autres protocoles comme le fait la classe 0, les passerelles pour les appareils de classe 1 vont traduire un protocole contraint dans son équivalent dans le monde non contraint.

Les dispositifs de classe 2 peuvent interagir directement avec d’autres nœuds sur l’internet, sans passer par une passerelle.

\begin{figure}[tbp]
\centerline{\includegraphics[width=1\columnwidth]{Pictures/Encpasul.png}}
\caption{Possibilités d'interconnection}
\label{fig-encap}
\end{figure}

  \vspace{2em}

\section{L'interopérabilité}

  \vspace{1em}


Un autre défi concerne le nombre de dispositifs. Certaines études prévoient 500 milliards de dispositifs à la fin de la décennie.

  \vspace{1em}
 \begin{wrapfigure}{r}{3cm}
\Youtube{https://youtu.be/2TQiTskfZtc}
\end{wrapfigure}
Avec cet énorme internet des objets, où presque chaque équipement comprendra des éléments de détection ou d’action, l’intégration dans un système d’information deviendra un véritable défi. Comme l’internet des objets actuel est conçu pour une verticale (c'est-à-dire pour des applications spécifiques), les dispositifs sont choisis et intégrés au moment de la conception du système. Les ingénieurs choisissent leurs capteurs, connaissent précisément leurs caractéristiques et écrivent leur code en fonction de ce qu'ils ont intégré. 

L’internet des objets massif change la donne. Les dispositifs ou les choses ne peuvent pas être intégrés dès le départ dans un système d’information statique. L’intégration doit se faire au fil du temps et doit gérer les évolutions des dispositifs pendant longtemps (les fabricants de dispositifs peuvent changer, les produits évolueront avec de nouvelles fonctionnalités, etc.)

Cette question d’interopérabilité a été formalisée dans le modèle \ac{LCIM}~\cite{tolk2003levels} (voir figure~\vref{fig-lcim}) peut être représenté par un compteur pour mesurer le degré d'interopérabilité.

\begin{figure}[tbp]
\centerline{\includegraphics[width=1\columnwidth]{Pictures/TPT2020.png}}
\caption{Niveau d'interopérabilité}
\label{fig-lcim}
\end{figure}


Il distingue six niveaux d’interopérabilité, parmi lesquels :

\begin{itemize}
     \item Au niveau zéro, on n'est pas connecté ; on ne parle à personne donc on n'a pas de problèmes d'interopérabilité. 
    \item  Au niveau 1, on est capable de transmettre de l'information,  mais il faut que les deux côtés connaissent les règles. On a un système intégré ; les applications doivent  connaître précisément les spécifications des objets avec lesquels ils communiquent,  car elles définissent ses propres formats des échanges de données. On pourrait prendre l'exemple qu'une carte électrique où un processeur communique avec des capteurs via un circuit imprimé. Le code tournant sur le processeur peut être écrit à l'avance car il y a peu de chance qu'un utilisateur dessoude les composants pour les remplacer par d'autres. Cela correspond à l'encapsulation, si l'on considère les objets en réseau de la figure~\vref{fig-encap}, l'élément d'interconnexion doit être configuré pour en fonction de l'objet émetteur encapsuler les données vers la bonne application chez le bon récepteur. 
    \item L’interopérabilité syntaxique (niveau 2) où deux nœuds peuvent échanger des données, sans être a préalable configurés pour cet échange. C'est le cas de l'Internet. En utilisant cette suite de protocoles, en ayant une adresse valide sur le réseau, toute application est capable d'échanger des données avec une autre. En revanche, les données que vous allez échanger sont propres à une application. Les vidéoconférences sont un très bon exemple d'interopérabilité de niveau 2. Vous ne pouvez pas utiliser Zoom si votre correspondant utilise Teams car les formats sont différents. L'\ac{IETF}  est le regroupement de différents acteurs (industriels, académiques,...) qui produisent les standards liés à ce réseau. Il se reconnaissent par l'acronyme \ac{RFC} suivi d'un nombre. 
    \item L’interopérabilité sémantique (niveau 3) implique que le récepteur est capable d'interpréter les données reçues.Le web est un très bon exemple d'interopérabilité de niveau 3. Quel que soit votre navigateur, vous pouvez afficher les pages d'un site web et suivre les liens. Le sens de l'information est compris de la même manière des deux côtés. Pour le Web, le format \ac{HTML} permet à l'aide de balises (mots-clés) de structurer un texte en y ajoutant des informations de formatage ou des liens vers d'autres documents. Le W3C définit les standards.
     \item Les niveaux supérieurs d'interopérabilité vont être lié à la précision du modèle qui va représenter le système.     
\end{itemize}


L'internet qu'on connaît est une très bonne illustration du besoin d'interopérabilite. Il a permis, grace à une uniformisation du réseau et une force baisse des coûts de transmission, de développer de nouveaux usages. Vous n'auriez jamais investi pour un réseau propre à la vidéoconférence et au télétravail. Mais en mutualisant les usages, c'est possible ! L'internet des objets doit suivre la même voie et également converger vers l'architecture de l'internet actuel. Le mot clé est donc l'interopérabilité.  
   \vspace{2em}

\section{Le besoin de standardisation}
  \vspace{1em}

Les objets n'ont pas attendu l'internet pour communiquer. Ils ont évolué chacun dans leur verticale, développant des solutions satisfaisantes mais limitées en termes d'évolution et d'interopérabilité. Cet état des lieux sur la dispersion des écosystèmes met en avant la nécessité :

\begin{itemize}
\item d’une coordination entre les alliances ;
\item d’une harmonisation ou d’un alignement des standards ;
\item de disposer au plus vite d’implémentations de référence et de modèles de référence.
\end{itemize}

Dans le cas contraire, l’interopérabilité ne pourra pas être traitée comme il se doit, les nouveaux services innovants multi-domaines ne seront pas couverts, et le développement de l’IoT pourrait être freiné voire avorté.

Les aspects liés à l’éducation et à la formation des acteurs de l’IoT, ainsi que ceux liés à l’acceptation, aux usages et évidemment au volet socio-économique de l’IoT, sont aussi des points essentiels dont il faut tenir compte.

Les organismes de normalisation tels que l’\ac{IETF} ou le \ac{W3C} ont conçu des protocoles ou des modèles de données capables de traiter l’interopérabilité. D’une certaine manière, c’est la clé du succès de l’internet actuel. Il a permis de résoudre le problème de l’interopérabilité au niveau syntaxique et sémantique mais au prix de messages volumineux.

Le défi pour l’internet des objets est de s’intégrer dans ce système distribué géant. Comme l’internet des objets est un nouveau venu, l’évolution devra se faire de son côté pour prendre en compte les règles existantes, mais en les adaptant. Les prochains chapitres traiteront de ces changements.

\section{Questions}

\Question{Nouveau Domaine}{Les objets communicants sont un tout nouveau domaine, lié aux progrès en miniaturisation des composants électroniques:
\begin{itemize}[label=$\circ$]
\item \Wrong{Vrai}
\item \Correct{Faux}
\end{itemize}
 }{Les objets communicants ont toujours existé, même avant le déploiement des protocoles de l'Internet. Ce qui est nouveau c'est leur intégration dans les architectures Internet actuelle pour mieux traiter l'information qu'ils produisent.
 }
 
 \Question{Protocoles}{Laquelle de ces affirmations est vraie ?
\begin{itemize}[label=$\circ$]
\item \Wrong{Il y a très peu de protocoles pour faire communiquer les Objets. Comme l'Internet est une technologie qui s'est très répondue, son succès va permettre aux objets de communiquer.}
\item \Correct{ Il y a beaucoup de solutions pour permettre à des objets de communiquer, l'Internet des Objets doit permettre de les fédérer.}
\end{itemize}
 }{Chaque métier a créé ses propres protocoles.}
 
 
\Question{Source de données}{Quelle est la principale source de création des données dans l'IoT ?
\begin{itemize}[label=$\circ$]
\item \Correct{les capteurs}
\item \Wrong{les nano-ordinateurs (type Raspberry Pi)}
\item \Wrong{Internet}
\item \Wrong{les serveurs Web}

\end{itemize}
 }{Les mesures effectuées par les capteurs constituent la principale source de création des données de l'IoT.}

\Question{Défis}{Quels sont les principaux défis technologiques pour l'Internet des Objets (3 réponses) ?
\begin{itemize}[label=$\square$]
\item \Correct{Avoir une consommation d'énergie faible.}
\item \Correct{Avoir une architecture protocolaire simplifiée.}
\item \Wrong{Pouvoir fonctionner sur des systèmes d'exploitation ouverts comme Linux.}
\item \Correct{Permettre de sécuriser les données qui peuvent être sensibles }
\item \Wrong{Transmettre en permanence leur état et des valeurs mesurées.}

\end{itemize}
 }{les objets sont généralement contraints en énergie et en capacité. Pour limiter la consommation d'énergie, il ne faut pas émettre ou recevoir tout le temps, donc la transmission en permanence des données ne va pas dans ce sens. De même, un système d'exploitation comme Linux apparait surdimensionné.}
 
\Input{Part02.0-ArchiIP}
\Input{Part02.5-Wireshark}
\Input{Part03.0-Modbus}
\Input{Part04.0-ArchiIoT}
\Input{Part05.0-Data}
\Input{Part06.0-VSensors}
\Input{Part06.5-beebotte}
\Input{Part07.0-LoPy}
\Input{Part08.0-Sigfox}
\Input{Part09.0-LoRaWAN}
\Input{Part10.0-CoAP}
\Input{Part10.5-aiocoap}
\Input{Part11.0-LwM2M}






%%%%%%%%%%%%%%%%%%%%%%%%%%%%%%%%%%%%%



\immediate\closeout\tempfile
\setboolean{Response}{false}

\cleardoublepage
\chapter{Réponses aux questions}
\input{questions}

\cleardoublepage
\printindex

\cleardoublepage
\printbibliography

\end{document}