%%%%%%%%%%%%%%%%%%%%%%%%%%%%%%%%%%%%%%%%%
%  My documentation report
%  Objetive: Explain what I did and how, so someone can continue with the investigation
%
% Important note:
% Chapter heading images should have a 2:1 width:height ratio,
% e.g. 920px width and 460px height.
%
%%%%%%%%%%%%%%%%%%%%%%%%%%%%%%%%%%%%%%%%%

%----------------------------------------------------------------------------------------
%	PACKAGES AND OTHER DOCUMENT CONFIGURATIONS
%----------------------------------------------------------------------------------------

\documentclass[11pt,fleqn]{book} % Default font size and left-justified equations

\usepackage[top=3cm,bottom=3cm,left=3.2cm,right=3.2cm,headsep=10pt,letterpaper]{geometry} % Page margins

\usepackage[dvipsnames]{xcolor}
\usepackage{lipsum} % Required for specifying colors by name
\definecolor{ocre}{RGB}{51,102,0} 
\definecolor{lightgray}{RGB}{229,229,229} 
% Font Settings
\usepackage{avant} % Use the Avantgarde font for headings
%\usepackage{times} % Use the Times font for headings
\usepackage{mathptmx} % Use the Adobe Times Roman as the default text font together with math symbols from the Sym­bol, Chancery and Com­puter Modern fonts

\usepackage{microtype} % Slightly tweak font spacing for aesthetics
\usepackage[utf8]{inputenc} % Required for including letters with accents
\usepackage[T1]{fontenc} % Use 8-bit encoding that has 256 glyphs

\usepackage[french]{varioref}
\usepackage[french]{babel}

\usepackage{wrapfig}
\usepackage{listings}
\usepackage{multicol}
\usepackage{multirow}
\usepackage{colortbl}
\usepackage{booktabs}
\newcommand{\tabitem}{~~\llap{\textbullet}~~}

\usepackage{soul}

\definecolor{deepblue}{rgb}{0.0,0.0,0.5}
\definecolor{deepred}{rgb}{0.6,0,0}
\definecolor{deepgreen}{rgb}{0,0.5,0}


\newcommand\pythonstyle{\lstset{
language=Python,
basicstyle=\ttfamily\footnotesize,
morekeywords={self},              % Add keywords here
frame=tb,                         % Any extra options here
showstringspaces=false
}}

\definecolor{backcolour}{rgb}{0.95,0.95,0.92}

\newcommand\termctyle{\lstset{
basicstyle=\ttfamily\footnotesize,
frame=tb,                         % Any extra options here
showstringspaces=false
}}


% Python environment
\lstnewenvironment{python}[1][]
{
\pythonstyle
\lstset{#1}
}
{}

\lstnewenvironment{termc}[1][]
{
\termcstyle
\lstset{#1}
}
{}


% RFC
\newcommand\rfc[1]{\href{http://www.ietf.org/rfc/rfc#1.txt}{\textcolor{blue}{RFC #1}\index{RFC #1}}}
\newcommand\pfunction[2]{\texttt{#2}}

\newcommand\glos[1]{\gls{#1}\index{#1}}

% QUESTION

\usepackage[most]{tcolorbox}
\usepackage{ifthen}

\provideboolean{Response}\setboolean{Response}{true}

\newcommand{\Correct}[1]{\ifthenelse{\boolean{Response}}{#1}{\textbf{#1}}}
\newcommand{\Wrong}[1]{\ifthenelse{\boolean{Response}}{#1}{\textcolor{black!20}{#1}}}

\newwrite\tempfile
\immediate\openout\tempfile=questions.tex


\newtcbtheorem[auto counter,number within=section]{theo}%
  {Question}{fonttitle=\bfseries\upshape, 
     arc=0mm, colback=blue!5!white,colframe=blue!75!black}{Question}
     
\newcommand\Question[3]{
\begin{theo}{#1}{summation}
#2
\immediate\write\tempfile{\noexpand\textbf{Question \thetcbcounter {} page \thepage} {} }
\immediate\write\tempfile{\unexpanded{#2}\noexpand\vspace{1em}\noexpand\newline}
\immediate\write\tempfile{\noexpand\textit{\unexpanded{#3}}\noexpand\newline\noexpand\newline}

\end{theo}
}

% MATHS PACKAGE
\usepackage{amsmath,tikz}
\usetikzlibrary{matrix}
\newcommand*{\horzbar}{\rule[0.05ex]{2.5ex}{0.5pt}}
\usepackage{calc}

% VERBATIM PACKAGE
\usepackage{verbatim}

\usepackage{tikz}

\usetikzlibrary{automata}
\usetikzlibrary[shadows]
\usetikzlibrary{shapes}
\usetikzlibrary[decorations.footprints] 
\usetikzlibrary{decorations.pathmorphing}
\usetikzlibrary{decorations.pathreplacing}
\usetikzlibrary{decorations.text}
\usetikzlibrary {arrows}
\usetikzlibrary{patterns}
\usetikzlibrary{calc}
\usetikzlibrary{external}

% Acronyms

\usepackage{makeidx}
\makeindex

\usepackage{acronym}

\let\oldac\ac
\renewcommand*{\ac}[1]{\oldac{#1}\index{#1}}

\newcommand\Index[1]{\textbf{#1}\index{#1}}

% Bibliography
\usepackage[style=alphabetic,sorting=nyt,sortcites=true,autopunct=true,babel=hyphen,hyperref=true,abbreviate=false,backref=true,backend=biber]{biblatex}
\addbibresource{bibliography.bib} % BibTeX bibliography file
\defbibheading{bibempty}{}

\input{structure} % Insert the commands.tex file which contains the majority of the structure behind the template

\begin{document}




\let\cleardoublepage\clearpage

%----------------------------------------------------------------------------------------
%	TITLE PAGE
%----------------------------------------------------------------------------------------

\begingroup
\thispagestyle{empty}
\AddToShipoutPicture*{\put(0,0){\includegraphics[scale=1.25]{v}}} % Image background
\centering
\vspace*{5cm}
\par\normalfont\fontsize{35}{35}\sffamily\selectfont
\textbf{PROGRAMMER L'INTERNET DES OBJETS }\\
{\LARGE }\par % Book title
\vspace*{1cm}
{\Huge Laurent TOUTAIN}\par % Author name
\endgroup

%----------------------------------------------------------------------------------------
%	COPYRIGHT PAGE
%----------------------------------------------------------------------------------------

\newpage
~\vfill
\thispagestyle{empty}

%\noindent Copyright \copyright\ 2014 Andrea Hidalgo\\ % Copyright notice

\noindent \textsc{IMT Atlantique}\\

\noindent Basé sur le MOOC PLIDO.\\ % License information

\noindent \textit{Publié le \today} % Printing/edition date

%----------------------------------------------------------------------------------------
%	TABLE OF CONTENTS
%----------------------------------------------------------------------------------------


\chapterimage{pano-5.jpg} % heading image

\pagestyle{empty} % No headers

\renewcommand\contentsname{Table des Matières}
\renewcommand{\bibname}{Bibliographie}
\tableofcontents% Print the table of contents itself

%\cleardoublepage % Forces the first chapter to start on an odd page so it's on the right

\pagestyle{fancy} % Print headers again

%----------------------------------------------------------------------------------------
%	CHAPTER 1
%----------------------------------------------------------------------------------------
\chapter*{Acronymes}
\begin{multicols}{2}
\begin{acronym}
\acro{3GPP}{3rd Generation Partnership Project}
\acro{ADSL}{Asymmetric Digital Subscriber Line}
\acro{CoAP}{Constrained Application Protocol}
\acro{CRC}{Cyclic Redundancy Check}
\acro{HTML}{HyperText Markup Language}
\acro{HTTP}{HyperText Transport Protocol}
\acro{HTTPS}{HyperText Transport Protocol Secure}
\acro{IBAN}{International Bank Account Num
ber}
\acro{IEEE}{Institute of Electrical and Electronics Engineers}
\acro{IETF}{Internet Engineering Task Force}
\acro{IoT}{Internet of Things}
\acro{IP}{Internet Protocol}
\acro{IPv4}{Internet Protocol version 4}
\acro{IPv6}{Internet Protocol version 6}
\acro{IRI}{International Resource Identifier}
\acro{ISBN}{International Standard Book Number}
\acro{ISO}{International Standardization Organization}
\acro{JSON}{JavaScript Object Notatin}
\acro{LCIM}{Levels of Conceptual Interoperability Model}
\acro{LNS}{LoRaWAN Network Server}
\acro{MQTT}{Message Queuing Telemetry Transport}
\acro{NAT}{Network Address Translation}
\acro{REST}{REpresentational State Transfer}
\acro{RFC}{Request For Comments}
\acro{RNIPP}{Répertoire National d'Identification des Personnes Physiques}
\acro{SCEF}{Service Capability Exposure Function}
\acro{STIC}{Sciences et Technologies de l’Information et de la Communication}
\acro{TCP}{Transmission Control Protocol}
\acro{TNT}{Télévision Numérique Terrestre}
\acro{UDP}{User Datagram Protocol}
\acro{URI}{Universal Resource Identifier}
\acro{URL}{Univeral Resource Locator}
\acro{URN}{Univeral Resource Name}
\acro{W3C}{World Wide Web Consortium}
\acro{WWW}{World Wide Web}
\acro{XML}{Extensible Markup Language}
\end{acronym}
\end{multicols}

%\input{Part1.0-Intro}
\newpage
%%----------------------------------------------------------------------------------------
%	CHAPTER 3
%----------------------------------------------------------------------------------------

\chapterimage{pano-tv1.png} % Chapter heading image
\acresetall
\chapter{ARCHITECTURE DE L'INTERNET}

\section{Les protocoles}
  
    \vspace{1em}

  \begin{wrapfigure}{r}{8.5cm}
\centerline{\includegraphics[width=.6\columnwidth]{Pictures/OSI.png}}
\caption{Extrait du standard ITU-T Rec. X.200 (1994 E)}
\end{wrapfigure}

Vous connaissez sûrement le principe d'empilement protocolaire dans les réseaux. Chaque protocole fournit un service et se base sur celui de la couche inférieure pour le réaliser. Le modèle d'origine définit sept couches pour transporter les données d'une application, n'importe où dans le monde.
  Les protocoles réseaux sont empilés les uns sur les autres, ceux du dessus utilisent les services offerts par ceux d’en dessous pour acheminer la donnée. Cela a donné lieu au modèle de référence de l’\ac{ISO} qui structure les réseaux depuis les années 1970. En théorie, il y a 7 \Index{couches}, mais l'internet a fait évoluer ce modèle et les numéros des couches, associés à des fonctionnalités, sont restés ; ce qui peut conduire à une numérotation étrange.
  
 \begin{figure}[tbp]
\centerline{\includegraphics[width=.5\columnwidth]{Pictures/hourglass.png}}
\caption{Empilement protocolaire de l'Internet}
\label{fig-hourglass}
\end{figure}

  \vspace{1em}

\begin{wrapfigure}{r}{3cm}
\Youtube{https://youtu.be/vQ7zMVrqHbA}
\end{wrapfigure}

  L'internet a simplifié cette architecture (cf. figure~\vref{fig-hourglass}). C'est pour ça que l'on retrouve moins de couches et que les numéros ne sont pas contigus. 
  
  
    \vspace{1em}

  Les deux premières couches  en partant du bas, regroupées sous le nom d'Interface, permettent de transmettre les données binaires sur un support physique. La couche 1 s’occupe de cette modulation sur un support physique particulier (fibre optique, paire de cuivre, onde radio). La couche 2 regroupe les mécanismes qui permettent de structurer cette donnée sous forme de bloc de taille finie appelées trames, de définir les méthodes d’accès, c’est-à-dire quand l’équipement peut émettre, et les formats des adresses utilisées pour identifier les équipements. Ethernet ou Wi-Fi sont des exemples de protocoles de niveau 2 (qui intègrent leur niveau 1).

\begin{itemize}
\item l’\ac{IEEE} qui propose des standards comme Ethernet pour les réseaux filaires ou Bluetooth et Wifi pour les réseaux radio,
\item le \ac{3GPP}  qui opère au même niveau et définit les protocoles pour la téléphonie cellulaire (4G),
\item \ldots
\end{itemize}

  
    \vspace{1em}

  
  
  Au-dessus, on a le protocole \ac{IP} standardisé par l'IETF.  Le protocole \ac{IP} s’adapte simplement à tout moyen de communication. \ac{IP} propose ainsi une abstraction des moyens de communication aux couches applicatives, rendant l’accès au réseau et l’adressage universels. Le traitement dans les \Index{routeur}s (équipements chargés d’aiguiller l’information dans le réseau) doit être le plus rapide possible pour traiter un maximum de paquets par seconde. De plus, \ac{IP} ne spécialise pas le réseau pour un service ou un autre ; il ne fait qu’aiguiller les paquets vers la bonne destination. Le réseau Internet est un réseau mondial construit autour de ce protocole permettant potentiellement d'atteindre tous les équipements qui y sont connectés. 
  
  
  Les experts de l'internet aiment cette représentation en \Index{sablier} où \ac{IP} apparaît en position centrale mais est plus petit comparé aux autres protocoles. Par conception, \ac{IP} est très simple ; à la fois pour être portés facilement sur de nombreux niveaux 2 et être facilement utilisable par les couches hautes, mais également pour traiter les données très rapidement dans les nœuds d'interconnexion. 
  
  \ac{IP} est mis en oeuvre partout sur Internet aussi bien dans les équipements en extrémité du réseau que dans les routeurs chargés d'envoyer les données vers la bonne destination.
  
  
    \vspace{1em}

  Au-dessus on trouve deux protocoles qui ne sont mis en oeuvre que dans les équipements d'extrémité.  Si le niveau 3 permet de joindre une machine, le niveau 4 va permettre d’identifier l'application qui doit traiter les données. Les "adresses" de ces applications sont des numéros compris entre 1 et 65535 appelés \Index{port}s. Par exemple, les serveurs Web utilisent le port numéro 80 ou le numéro 443. 
  
  
  Le protocole \ac{TCP} va surveiller les données transférées et sera capable de retransmettre des données perdues, ralentir ou accélérer le transfert de données s’il détecte une saturation du réseau. En revanche, sa mise en œuvre est complexe et coûteuse en mémoire. Dans les cas simple, \ac{UDP} est préféré ; il n'apporte pas de traitement supplémentaire \ac{UDP}, c'est un protocole minimal qui se contentent d'aiguiller les données vers la bonne application sans aucun autre contrôle.
  

    \vspace{1em}

  
  Au-dessus, on trouve les applications qu'historiquement on classe dans la couche 7. Les applications sont très nombreuses mais la plus répandue est \ac{HTTP} qui sert à transporter les pages web, mais également elle permet des communications directes entre ordinateurs. 
  
  
    \vspace{1em}


   \begin{figure}[tbp]
\centerline{\includegraphics[width=1\columnwidth]{Pictures/fullIPstack.png}}
\caption{Principaux protocoles de l'Internet}
\label{fig-fullstack}
\end{figure}

  Pour le grand public, l’internet désigne surtout la totalité de cet assemblage protocolaire et est souvent confondu avec l’application qui a démocratisé son usage : le Web.  C'est vrai également pour les techniciens, le trafic produit par le Web est largement présent dans l'Internet. Ce schéma, figure~\vref{fig-fullstack} reprend la pile protocolaire majoritairement utilisé dans l'internet. On voit qu'au niveau 3 on a deux versions du protocole \ac{IP} ; la version 4 est la version historiquement déployée et elle a eu tellement de succès qui est de plus en plus difficile d'avoir des adresses \ac{IPv4} pour les machines. Pour permettre au réseau de continuer de fonctionner, une nouvelle version a été développée. \ac{IPv6} rend l'adressage quasi infini avec des adresses sur 128 bits. \ac{IPv6} gagne petit à petit du terrain dans les usages classiques et c'est surtout une brique essentielle pour l'internet des objets. 
  
  Le Web utilise majoritairement le protocole \ac{HTTP}. Et comme \ac{HTTP} repose sur \ac{TCP}, ces deux protocoles sont dominants sur le réseau. 
  
  Finalement ce graphique ajoute une couche supplémentaire, au-dessus de la couche 7, pour indiquer comment les données transportées sont structurées avec des formats comme \ac{XML} ou \ac{JSON} que nous verrons dans la suite.
  
  
  \Question{Pile Protocolaire}
  {Dans la pile protocolaire de l'internet, quels protocoles ont pour fonction d'aiguiller les paquets jusqu'à leur destination (2 réponses) 
  \begin{multicols}{4}
  \begin{itemize}[label=$\square$]
   \item \Wrong{Ethernet}
   \item \Wrong{IEEE}
   \item \Wrong{802.15.5}
   \item \Wrong{Wi-Fi}
   \item \Correct{IPv4}
   \item \Correct{IPv6}
   \item \Wrong{UDP}
   \item \Wrong{TCP}
   \item \Wrong{MQTT} 
   \item \Wrong{HTTP}
   \item \Wrong{CoAP}
   \item \Wrong{XML}
   \item \Wrong{JSON}
  \end{itemize}
  \end{multicols}
  }
  {Le protocole \ac{IP} permet de transporter l'information d'un bout à l'autre du réseau en utilisant les adresses \ac{IP} contenues dans les paquets. Il existe deux versions de ce protocole \ac{IPv4} initialement déployé et \ac{IPv6} qui offre beaucoup plus de capacité d'adressage.}
  
  \section{Les fondements du Web}
  
    \vspace{1em}
\begin{wrapfigure}{r}{3cm}
\Youtube{https://youtu.be/PKKzV-Vy33s}
\end{wrapfigure}

  L'architecture qui a conduit au Web est une formidable source d’inspiration pour le développement de nouveaux services car il est l’un des plus grands succès reposant sur le réseau Internet. Le Web forme de grands systèmes distribués et repose sur plusieurs principes qui le rendent universel et évolutif. La navigation avec un navigateur n’est que la partie visible du trafic ; les principes du web sont également utilisés pour le streaming vidéo, les échanges entre ordinateurs. 
  
  Le Web et ses extensions sont basés sur un modèle client-serveur. Les serveurs possèdent des ressources et les clients peuvent y accéder ou les modifier grâce à un protocole tel que \ac{HTTP}. Le modèle client-serveur est quelque chose de courant dans les réseaux informatiques, mais le Web suit certaines directives de conception connues sous le nom de \ac{REST}.

    \vspace{1em}

Selon Roy Fielding, qui a défini ce modèle~\cite{rest}, \ac{REST} est un ensemble de principes, de propriétés et de contraintes. \ac{REST} utilise le modèle de communication client-serveur et utilise généralement le protocole \ac{HTTP}.

Le principe \ac{REST} permet de concevoir des serveurs évolutifs. Un serveur doit être sans état, ce qui signifie qu’il ne conserve pas d’information après avoir répondu à une demande d’un client. Cela permet de simplifier le traitement dans le serveur qui doit traiter les requêtes d'un grand nombre de clients. 

Cela impose que l’état soit situé du côté du client. Cet état est alimenté à partir des données structurées que le client reçoit du serveur. Ainsi, lorsqu’un client demande une page Web, celle-ci peut contenir d’autres \ac{URI} pour la compléter, par exemple des images, des feuilles de style, des scripts, etc.

Le client doit donc comprendre les données que le serveur lui envoie et donc connaître le format de représentation de la ressource qu'il reçoit pour y retrouver les \ac{URI}. Donc, en plus de la ressource elle-même, le serveur ajoute des informations complémentaires, appelées métadonnées. Elles intègrent entre autres le format du contenu (content format). Il peut s’agir de texte pur, d’une image ou d’un format de texte structuré tel que \ac{HTML} ou \ac{JSON}.
  
      \vspace{1em}

\subsection{Ressources}

    \vspace{1em}

   \begin{wrapfigure}{r}{9cm}
\centerline{\includegraphics[width=.6\columnwidth]{web.png}}
\end{wrapfigure}
   L'élément de base est la ressource que l'on peut définir comme un bloc de données de taille finie. Les ressources peuvent elles-mêmes contenir des références à d'autres ressources qui à leur tour vont faire référence à d'autres ressources etc. etc. Cela forme un maillage entre ressources qui est comparé à une toile d'araignée (web en anglais). La ressource peut être, par exemple, une image dans ce cas elle ne fera pas référence à autre chose. Pour faire référence à une autre ressource, son contenu doit être structuré et doit donc être défini dans un format où l'on peut facilement comprendre qu'une partie du contenu est une référence a une autre ressource. HTML est un de ces langages qui permet aux pages web de se référencer entre elles par le biais de liens. 
  
      \vspace{1em}

\subsection{identifiants} 

    \vspace{1em}

   
   Chaque ressource du Web est identifiée par une valeur unique appelée \ac{URI}. Si l’URI contient des caractères internationaux, (comme les lettres accentuées, ...)   il est appelé \ac{IRI}.

Les \ac{URI} permettent de désigner une ressource de manière non ambiguë, c'est-à-dire que l'on ne retrouvera pas le même \ac{URI} pour désigner deux ressources différentes.  Par construction, la structure de l’\ac{URI} est hiérarchique, ce qui permet de créer des identificateurs uniques de manière distribuée. Si vous voulez identifier une ressource, vous devez posséder une séquence unique : un numéro de téléphone, un numéro de sécurité sociale, un nom de domaine. En y ajoutant quelque chose d'unique pour nous, cela crée un identifiant globalement unique. 

Par exemple, pour identifier une image, on peut la nommer

\begin{lstlisting}[backgroundcolor = \color{yellow!20}]
image
\end{lstlisting}

\noindent

mais il y a peu de chance que ce nom soit unique, d'autres personnes sur Terre ont sûrement eu la même idée. En revanche, si je la fais précéder de mon numéro de téléphone

\begin{lstlisting}[backgroundcolor = \color{yellow!20}, 
                    basicstyle= \ttfamily]
33667789078image
\end{lstlisting}

\noindent
 sera unique si je ne nomme qu'une seule ressource "image". Un autre utilisateur sur le même principe pourra nommer sa ressource :

\begin{lstlisting}[backgroundcolor = \color{yellow!20}, 
                    basicstyle= \ttfamily]
33667239018image
\end{lstlisting}

\noindent
sans ambiguïté possible. Cependant, comme le numéro de téléphone est unique dans l'espace des numéros de téléphone, d'autres numéros uniques pourraient entrer en conflit dans d'autres espaces de numérotation. 

Pour éviter les conflits, il est intéressant de donner, au début l'espace de numérotation, par exemple :

\begin{lstlisting}[backgroundcolor = \color{yellow!20}, 
                    basicstyle= \ttfamily]
tel:33667789078image
\end{lstlisting}

\noindent
et

\begin{lstlisting}[backgroundcolor = \color{yellow!20}, 
                    basicstyle= \ttfamily]
ss:33667789078image
\end{lstlisting}

\noindent
Les deux identifiants seront uniques, même si le hasard a fait que ce numéro de téléphone et ce numéro de sécurité sociale coïncident.

      \vspace{1em}


Les \ac{URI} formalisent ce principe. Le \rfc{3986} explique comment ils peuvent être construits. Un URI commence par un schéma indiquant l’autorité de nommage, suivi d’une valeur d’autorité puis d’un chemin dans l’espace d’autorité. Des caractères comme les ":" ou les "/" sont utilisés pour améliorer la lisibilité de l'\ac{URI}.



Par exemple~:

\begin{lstlisting}[backgroundcolor = \color{yellow!20}, 
                    basicstyle= \ttfamily]
mailto:mduerst@ifi.unizh.ch
ssh://utilisateur@example.com
ftp://ftp.is.co.za/rfc/rfc1808.txt
\end{lstlisting}

      \vspace{1em}

Ainsi, si je mets une ressource sur un site Web, celui-ci est identifié par un nom de domaine, par exemple \texttt{example.com}. Je suis propriétaire de ce nom. Je peux donc l’utiliser pour identifier de manière unique ma ressource. Si on reprend le principe de construction d’un URI, j’aurai :

\begin{lstlisting}[backgroundcolor = \color{yellow!20}, 
                    basicstyle= \ttfamily]
http://example.com/ma_ressource
\end{lstlisting}

Personne d’autre dans l’univers ne pourra identifier ses ressources avec cette chaîne de caractères puisque \texttt{example.com} m’appartient. Je dispose donc d’un espace de nommage infini qui me permet de désigner l’ensemble infini de ressources sans que personne d’autre ne puisse prendre les mêmes noms. Un \ac{URI} est une construction administrative permettant d’attribuer un identifiant unique global à une ressource spécifique.


  \begin{figure}[tbp]
\centerline{\includegraphics[width=1\columnwidth]{Pictures/Capture15.png}}
\caption{Structuration d'une URI}
\label{fig-stucURI}
\end{figure}

L’\ac{URI} (cf.figure~\vref{fig-stucURI}) a pour but de facilement nommer une ressource, de pouvoir lier les ressources entre elles pour former cette toile d’araignée mondiale. Le schéma définit à la fois l'espace de nommage de l'autorité et son format. Une adresse \ac{IP} ou un nom de domaine comme autorité est à la fois un moyen d'assurer l'unicité globale, mais également de savoir comment accéder à la ressource. 

      \vspace{1em}

   Par exemple, \Index{spotify} a défini son propre schéma et ensuite il n'a plus besoin d'autorité mais structure le chemin pour référencer une playlist. 
   
   
   
   Tous les livres ont un numéro \ac{ISBN} qui permet d'identifier. Il peut être intégré aussi dans une URI. Il faut voir que ces deux types d'identifiants permettent de référencer un objet unique mais rien qu'en le lisant on ne peut pas accéder à la ressource. On appelle cette sous famille des \ac{URI}, des \ac{URN}. Si en revanche, en lisant l'URI on peut localiser la ressource on a une \ac{URL}.

Un sous-ensemble d’\ac{URI} peut être directement utilisé pour localiser la ressource, c’est-à-dire trouver sur quel serveur se trouve la ressource et comment y accéder. Il s’agit d’une URL bien connue du grand public et utilisée par les navigateurs Web.

Le schéma http est bien pratique car il peut se lire également comme un \ac{URL}. Ce schéma donne~:

\begin{itemize}
\item le protocole à utiliser pour accéder à la ressource (http),
\item l’autorité qui indique l’adresse du serveur (et son port),
\item et enfin, le chemin d’accès de ce que l'on va demander au serveur et qui peut parfois correspondre à une arborescence de fichiers sur un serveur.
\end{itemize}
   
   
       \vspace{1em}

  Mais il faut bien voir que le but initial est de faire un identifiant unique. Le schéma https donne la manière dont sera construit la suite et dans un second temps uniquement sera vu comme le protocole à utiliser pour accéder à la ressource. L'autorité est unique et dans un second temps servira à localiser le serveur. Et finalement le chemin va indiquer comment parvenir à accéder à la ressource sur le serveur. Donc les ressources de notre toile d'araignée mondiale sont présentes sur des serveurs et chaque ressource possède un identifiant unique. Dans un premier temps, le client connaît l'\ac{URI} d'une ressource. Si c'est une \ac{URL}, il peut contacter le serveur. Le serveur lui retourne la ressource. Le client l'analyse et découvre les \ac{URI} qu'il contient. Il peut donc interroger le ou les autres serveurs pour reconstruire localement une partie de la toile nécessaires au traitement que le client veut effectuer.
  
         \vspace{2em}

\Question{Unicité}{Qu'est ce qui est unique dans le monde (6 réponses) ?
\begin{itemize}[label=$\square$]
\item \Wrong{Un prénom.}
\item \Wrong{Un nom de famille.}
\item \Correct{un numéro de sécurité sociale utilisé en France.}
\item \Correct{un numéro de passeport.}
\item \Correct{un numéro de téléphone portable avec son préfixe international.}
\item \Correct{un numéro complet de compte en banque (\ac{IBAN}).}
\item \Wrong{l'adresse IP de ma machine dans mon réseau privé.}
\item \Correct{l'adresse IP d'un serveur fun-mooc (\texttt{51.255.9.16}).}
\item \Correct{le nom de domaine \texttt{plido.net}.}
\item \Wrong{le nom d'une ville.}
\end{itemize}}
{Ni le prénom, ni le nom de famille, ni leur combinaison ne forment des séquences uniques comme le dirait Jean Dupont.
Les numéros de passeport, de sécurité sociale, de téléphone, de compte bancaire sont par construction uniques dans leur espace, mais il pourrait y avoir des recouvrements. C'est pour cela qu'il faut indiquer la source ou l'autorité qui l'a alloué pour garantir l'unicité. Pour le passeport, l’autorité est le pays qui l’a produit. Le numéro de sécurité sociale correspond au numéro d'inscription au  \ac{RNIPP} (cf. \url{https://www.service-public.fr/particuliers/vosdroits/F33078}). Le code du pays pour le numéro de téléphone est attribué par l'UIT puis chaque opérateur a sa zone de numérotation. Pour le compte bancaire, c’est bien sûr la banque ; chaque banque ayant son propre code contenu au début de l'\ac{IBAN}.
L'adresse IP dans un réseau privé n'est pas unique. Le \rfc{1918} définit des plages d'adresses que tout le monde peut utiliser localement. Pour sortir sur Internet, il faut un dispositif spécial, appelé \ac{NAT}, qui va convertir l'adresse privée en une adresse publique qui, elle, est unique dans le monde. Les serveurs doivent être dans cet espace public et ont donc une adresse unique dans le monde.
Les noms de domaines sont uniques dans le monde par construction mais il peuvent être partagés par plusieurs utilisateurs. On peut donc les étendre comme \texttt{machine1.plido.net}, \texttt{machine2.plido.net}... ou, s'il s'agit de la même machine, ajouter un numéro de port après pour indiquer le service : \texttt{www.plido.net:80}, \texttt{www.plido.net:8080}.
Quant au nom d'une ville, il n'est pas unique. Il faut aussi préciser le pays, voire la région.
}
 

  \subsection{Interactions}
  
         \vspace{1em}

  Les interactions entre clients et serveurs sont très simples. Le client va gérer les interactions avec les ressources sur un serveur. Il peut, par exemple, récupérer une ressource grâce à une méthode \Index{GET}. Il peut également écrire les données dans une ressource existante grâce à une méthode \Index{PUT}. 
  
  Le nombre d'interactions est très limité. \ac{HTTP} ou \ac{HTTPS} est un moyen de mettre en oeuvre ces méthodes. 




\ac{HTTP} est un protocole qui peut être utilisé pour mettre en œuvre un serveur Web les principes de \ac{REST} (qualifié en anglais de \Index{RESTfull}). \ac{HTTP} définit différentes méthodes permettant au client d’interagir avec les ressources sur le serveur~:

\begin{itemize}
    \item \Index{GET} est utilisée pour récupérer la représentation d’une ressource (par exemple page Web, valeur de température d’un capteur, etc.). Par exemple, la figure~\vref{fig-GET} donne le format d’en-tête \ac{HTTP} GET pour récupérer une page Web :

 \begin{figure}[tbp]
\centerline{\includegraphics[width=1\columnwidth]{Pictures/GET.png}}
\caption{Contenu d'une requête HTTP GET}
\label{fig-GET}
\end{figure}

\item  \Index{HEAD} est utilisée pour récupérer uniquement les métadonnées présentes dans les en-têtes de réponse sans le corps de réponse ;
\item  \Index{POST} est utilisée pour indiquer au serveur une nouvelle ressource  ;
\item  \Index{PUT} est utilisée pour stocker une ressource à l’endroit identifié par l’\ac{URI} dans la requête. Si la ressource existe déjà, elle sera modifiée ;
\item \Index{PATCH} permet au client de ne modifier qu’une partie de la ressource ;
\item  \Index{DELETE} est utilisée pour supprimer la ressource définie.


\end{itemize}

 \Question{Etat}{Est-ce que le serveur garde un état des précédentes requêtes ?
\begin{itemize}[label=$\circ$]
\item \Wrong{Oui}
\item \Correct{Non}
\end{itemize}}
{Le serveur répond à une requête puis passe à la suivante. Il ne garde pas d'état. En revanche, une requête peut servir à modifier le contenu d'une ressource.}

\Question{Wold Wide Web}{Le World Wide Web est basé sur ce principe des états pour :
\begin{itemize}[label=$\circ$]
\item \Wrong{fonctionner à la fois sur des ordinateurs et des téléphones portables,}
\item \Correct{pouvoir servir un grand nombre de requêtes}

\item \Wrong{chiffrer les communications,}
\end{itemize}}
{Voir le commentaire précédent}

 \Question{Repésentation de l'Information}
  {Quels sont les formats qui permettent de représenter des informations structurées (2 réponses)~: 
  \begin{multicols}{4}
  \begin{itemize}[label=$\square$]
   \item \Wrong{Ethernet}
   \item \Wrong{IEEE}
   \item \Wrong{802.15.5}
   \item \Wrong{Wi-Fi}
   \item \Wrong{IPv4}
   \item \Wrong{IPv6}
   \item \Wrong{UDP}
   \item \Wrong{TCP}
   \item \Wrong{MQTT} 
   \item \Wrong{HTTP}
   \item \Wrong{CoAP}
   \item \Correct{XML}
   \item \Correct{JSON}
  \end{itemize}
  \end{multicols}
  }
  {Il s'agit de XML et JSON qui permettent d'envoyer des données stucturées. Les autres propositions indiquent des protocoles de transport de l'information de niveau 2, 3, 4 et 7.}
  
\Question{Schéma}{Dans l'URI \texttt{https://plido.net/unit/definition.html}, quel est le schéma ?\newline}
{\noindent\texttt{http}~: le Schéma indique comment sera construite l'URI}

\Question{Authorité}{Dans l'URI \texttt{https://plido.net/unit/definition.html}, quel est l'autorité ?\newline}
{\noindent\texttt{plido.net}~: il s'agit de la séquence globalement unique.}


\section {Modèle Publish/Subscribe}

Il existe d’autres formalismes que REST. Un autre formalisme, très populaire, est orienté "diffusion" en utilisant le principe ”publier/abonner” ou ”publish/subscribe”. Comme nous allons le montrer dans la suite, même si les fonctionnalités entre ces deux modes peuvent sembler similaires, la philosophie de conception est très différente : publish/subscribe vise des applications intégrées tandis que REST vise l’interopérabilité globale. 

Le modèle publish/subscribe fait le découplage entre l’expéditeur d’un message et son destinataire. Dans ce paradigme (cf. figure ci-dessous), il existe des ”Publishers” qui produisent des données ou des messages et envoient le message à une entité généralement appelée ”Broker”. En outre, les messages peuvent être classés en ”Topics”, contenus ou types, etc. Ensuite, il existe des abonnés qui souscrivent au broker, par exemple à un topic donné, afin de recevoir les messages qui les intéressent, comme montré dans le schéma.

Le broker peut alors utiliser des filtres pour envoyer uniquement ces messages aux abonnés du topic concerné. Il existe plusieurs protocoles Publish-Subscribe tels que MQTT (Message Queuing Telemetry Transport), AMQP (Advanced Message Queuing Protocol), JMS (Java Messaging Service) ou XMPP (Extensible Messaging Protocol et Présence).

\subsection{MQTT}

MQTT est détaillé dans la suite du cours car il est très populaire pour la communication entre processus, mais également dans l’internet des objets.

Imaginons par exemple que plusieurs capteurs soient installés dans deux bâtiments A et B. Certains capteurs collectent des informations sur la température et d’autres collectent des informations sur l’humidité. Ces capteurs peuvent envoyer les données régulièrement à un broker central.

Les données peuvent être classées en différentes rubriques qui peuvent également être organisées de manière hiérarchique. Par exemple, le topic /sensor signifie "toutes les données de capteurs", \texttt{/sensor/buildingA/} signifie "des données de capteurs uniquement installées dans le bâtiment A". En plus, \texttt{/sensor/buildingA/temperature} pourrait signifier "des données de capteurs de température installés uniquement dans le bâtiment A".

Certains abonnés peuvent s’abonner aux messages en fonction de leur intérêt. Ainsi, un abonné intéressé uniquement par les données d’humidité du bâtiment B peut s’abonner au sujet \texttt{/sensor/buildingB/humidity} et le broker n’enverra que ces données à cet abonné.

\subsection{différence avec REST}

Les principaux avantages du paradigme publish-subscribe par rapport au paradigme client-serveur, tels qu'inclus en REST, sont les suivants :
\begin{itemize}
\item faible couplage entre émetteur et récepteur, le broker sert d'intermédiaire et stocke les informations ;
\item passage à l’échelle. Les données provenant d'une source ne sont émises qu'une fois par la source. Le broker les recopie vers tous les abonnés. Dans un mode client/serveur, les données doivent être émises par le serveur autant que fois que les clients le demandent.
\end{itemize}
L’absence de couplage entre l’expéditeur et le destinataire se fait en termes d’espace, de temps et de synchronisation. Celui qui publie les données a une tâche simplifiée. Il n'a pas à gérer ou connaître ceux qui les consomment, il n'a qu’à les envoyer au broker.

MQTT est très léger et conçu pour les périphériques de faibles puissances. Il a une très petite empreinte logicielle et est optimisé pour fonctionner dans les environnements à faible bande passante. Cela rend MQTT idéal pour les applications IoT. Malgré tout, l’usage de TCP et des très nombreux acquittements peut s’avérer lourds pour les équipements ou les réseaux très contraints. Une version plus légère basée sur UDP existe pour ces cas d’usage, mais elle est peu utilisée.

S'ils se ressemblent, les principes de nommage des topics MQTT et des URI REST sont complètement différents. Par rapport à MQTT, le chemin dans l'URI n'a pas de sémantique. Il a juste vocation à être unique. Il ne peut pas être utilisé pour agréger plusieurs sources d'information. Si deux capteurs publient respectivement sur les topics \texttt{/sensor/buildingA/temperature} et \texttt{/sensor/buildingB/temperature}, un subscriber peut s'abonner au topic \texttt{/sensor/*/temperature} pour recevoir toutes les mesures ; ce qui est impossible avec REST : il faudra autant de requêtes que de capteurs pour récupérer l'ensemble des mesures.

Les URI sont simplement uniques au monde par leur construction alors que les topics du MQTT sont spécifiques à une application. Un topic MQTT peut être interprété différemment par deux applications différentes. Cela ne permet pas une interopérabilité sémantique. Les abonnés doivent être construits avec une connaissance des topics utilisés par les publieurs.

\newpage



%%%%%%%%%%%%%%%%%%%%%%%%%%%%%%%%%%%%%%%%%%%%%%%

% ModBus

%%%%%%%%%%%%%%%%%%%%%%%%%%%%%%%%%%%%%%%%%%%%%%%

\chapterimage{pano-tv1.png} % Chapter heading image

\chapter{Modbus}

\section{Introduction}
  
 \Index{Modbus} est apparu en 1979 à une époque où l'internet n'existait pas encore ! Il est toujours très populaire dans l'industrie. A l'origine Modbus était construit sur un bus série \Index{RS-485} qui connectait différents équipements  appelé (cf. figure~\vref{fig-modbus1})~:
 \begin{itemize}
 \item secondaires ou slaves et 
 \item un primaire appelée aussi master qui gère les communications. 
 \end{itemize}
 
 
 
 \begin{figure}[tbp]
\centerline{\includegraphics[width=1\columnwidth]{Pictures/Capture35.png}}
\caption{Architecture filaire de Modbus}
\label{fig-modbus1}
\end{figure}

 
 
 Chaque secondaire à un numéro unique ou adresse. Les adresses sont comprises entre 1 et 247. Le primaire n'a pas besoin d'une adresse puisque toutes les communications ont lieu avec lui. 
 
   \vspace{1em}


 Le primaire envoie une requête à un secondaire et le secondaire répond au primaire. Les communications directes entre deux secondaires ne sont pas possibles. 
 
 \subsection{Registres}
 
 
 Un équipement de Modbus peut prendre deux choses à travers des registres : 
 \begin{itemize}
     \item les relais qui peuvent prendre une valeur binaire "on" ou "off". Si le primaire peut modifier l'état et, bien sûr le lire, c'est appeler un \textit{\Index{coil}}. Si la valeur binaire peut uniquement être lu c'est un \textit{\Index{discrete input}}.
     \item des registres sur 16 bits. Ils sont utilisés pour représenter une valeur comme un courant électrique, une température, une vitesse de rotation,... De même, si on peut uniquement lire la valeur à est appelée un \textit{\Index{input register}} sinon, si elle peut être également être modifiée par le primaire, elle est appelée un \textit{\Index{holding register}}.
 \end{itemize}
 
    \vspace{1em}

 
Un équipement Modbus peut avoir jusqu'à 10~000 registres de ces quatre catégories. 

\subsection{Protocole}

Modbus est un protocole requête/réponse. Le primaire envoie une requête à l'adresse d'un équipement pour lire ou écrire un de ces registres. 
\begin{figure}[tbp]
\centerline{\includegraphics[width=1\columnwidth]{Pictures/Capture37.png}}
\caption{Trame Modbus}
\label{fig-modbus2}
\end{figure}

 

Une trame Modbus est une séquence de caractères commençant par un octet avec l'adresse du secondaire suivi d'une commande ou code de fonctions spécifique à chaque catégorie de registre~:
\begin{itemize}
    \item 1 pour lire un coil,
    \item 2 pour lire un discrete input,
    \item 3 pour lire un holding register,
    \item 4 pour lire un input register,
    \item 5 pour écrire un coil,
    \item 6 pour écrire un holding register,
\end{itemize}

    \vspace{1em}

La suite de la trame contient les données puis un \ac{CRC} pour valider qu'il n'y a pas d'erreur de transmission dans la trame. La partie donnée peut être différente dans la requête et la réponse. Par exemple pour lire un holding register, la requête contient l'adresse du premier registre à lire et le nombre de registres à lire et la réponse contient le nombre de données transmises suivi de leurs valeurs. Pour écrire sur un registre, les données de la trame seront l'adresse du registre et les données à écrire.

\subsection{Exemple: XY-MD02}


Regardons de plus près un exemple concret. On va utiliser un capteur de température et d'humidité, le \Index{XY-MD02} (cf. figure~\vref{fig-XYMD02}) dont les spécifications sont facilement accessibles via une recherche sur Internet. 

\begin{figure}[tbp]
\centerline{\includegraphics[width=0.5\columnwidth]{Pictures/XY-MD02.png}}
\caption{XY-MD02}
\label{fig-XYMD02}
\end{figure}

La partie verte, se compose de quatre connecteurs dont la signification est indiquée sur l'étiquette. Les deux bornes de gauche constituent le bus \Index{RS-486} nommées \texttt{A+} et \texttt{B-} et les deux bornes de droites permettent d'alimenter électriquement l'équipement avec une tension comprise entre 5V et 30V.

    \vspace{1em}


Un adaptateur \Index{USB}/\Index{RS-486} (cf. figure~\vref{fig-USBRS486}) est connecté à un ordinateur. On y retrouve les deux bornes A+ et B- du bus RS-846. L'ordinateur joue le rôle de primaire qui va interroger le capteur de température.  

\begin{figure}[tbp]
\centerline{\includegraphics[width=0.3\columnwidth]{Pictures/rs485-usb.png}}
\caption{Adaptateur USB/RS-486}
\label{fig-USBRS486}
\end{figure}

    \vspace{1em}

Le programme \Index{QModMaster}\footnote{\url{https://sourceforge.net/projects/qmodmaster/}} (cf. figure~\vref{fig-qmodmaster}) permet d'interroger ou d'écrire les registres des secondaires. Dans la fenêtre de gauche permet d'accèder aux registres des secondaires. La fenêtre de droite montre le trafic ayant circulé sur le bus RS-486.

\begin{figure}[tbp]
\centerline{\includegraphics[width=1\columnwidth]{Pictures/modbus-trace.png}}
\caption{QModMaster avec trace des messages}
\label{fig-qmodmaster}
\end{figure}

Pour que le primaire puisse se connecter au secondaire, en plus du nom du port série (ici \texttt{COM3}), il faut disposer de plusieurs informations que l'on peut retrouver dans sa documentation :

\begin{itemize}
    \item la vitesse de transmission sur le bus (ici 9~600 bit/s) et le codage des caractères transmis (ici 8 bits sans bit de parité et un bit de stop)\footnote{\url{https://en.wikipedia.org/wiki/8-N-1}}.
    \item l'adresse du secondaire sur le bus.
\end{itemize}

    \vspace{1em}

La documentation donne également la nature des registres et leur codage. Le tableau~\vref{tab-XY-IR} reprend la définiton des \textit{Input Registers}. Il s'agit de registres qui ne peuvent qu'être lus. La spécification indique que la température est stockée dans le registre 1 sur une longueur de 2 octets, soit l'intégralité de celui-ci.

\begin{table}
\begin{center}
\begin{tabular}{|c|c|c|c|}
\hline
 \rowcolor{purple!10} Register Type & Register Address & Register Contents & Bytes \\ \hline \hline
 \multirow{2}{*}{Input register} & 0x0001 & Temperature & 2 \\ \cline{2-4}
                                 & 0x0002 & Humidity & 2 \\  \hline
\end{tabular}
\end{center}
\caption{Input Register d'un XY-MD02}
\label{tab-XY-IR}
\end{table}

    \vspace{1em}

La documentation indique que le secondaire a l'adresse 0x01 sur le bus RS-486. Il ne reste plus qu'à y envoyer une requête Modbus pour lire ce registre. La fenêtre de trace à droite sur la figure~\vref{fig-qmodmaster} donne les échanges. Nous allons analyser les deux dernières lignes. La première indique le contenu de la requête et la dernière la réponse du secondaire~:

\begin{termc}[backgroundcolor=\color{backcolour}, escapechar=@]
@\texttt{01 \textcolor{blue}{04} \textcolor{purple}{00 01} \textcolor{green!60!black}{00 01} \textcolor{black!30}{60 0A}}@
@\texttt{01 \textcolor{blue}{04} \textcolor{orange}{02} \textbf{00 D5} \textcolor{black!30}{78 AF}}@
\end{termc}

La requête commence par l'adresse du secondaire (\texttt{01}), puis par l'action (\texttt{\textcolor{blue}{04}}) pour lire un \textit{Input Register}, suivit de l'adresse du registre (\texttt{\textcolor{purple}{00 01}}) et du nombre de registres à lire (\textcolor{green!60!black}{00 01}). La requête se termine par la \ac{CRC} validant l'intégrité de la trame (\texttt{\textcolor{black!30}{60 0A}}).

la réponse contient également l'adresse du secondaire (\texttt{01}) et l'action, suivi de la taille de la réponse en octets (\texttt{\textcolor{orange}{02}}) et du résultat demandé (\texttt{\textbf{00 D5}}). 

\Question{Humidité}
{En regardant les échanges de la figure~\vref{fig-qmodmaster} quelle est la valeur mesurée pour l'humidité ?}
{Seul le premier échange demande la lecture de 2 registres à partir du registre ayant l'adresse 0x0001. Dans la réponse on obtient la valeur des deux registres consécutif (\texttt{00 DA} et \texttt{01 D4}). Le tableau~\vref{tab-XY-IR} indique que l'humidité est le second registre, on a donc \texttt{01 D4}.}

Reste à pouvoir interpréter cette valeur. La documentation indique que la valeur est en dixièmes de degrés et que l'unité est le Celsius. En convertissant \texttt{\textbf{00 D5}} en décimal, on obtient 213, soit 21.3\textcelsius.

\Question{Évolution des températures}
{En regardant les échanges de la figure~\vref{fig-qmodmaster} quelle est l'évolution de la température au cours du temps ?}
{On trouve 3 valeurs mesurées à 19:11:48, 20:02:39 et 20:02:44~: \texttt{00 DA}, \texttt{00 D5} et \texttt{00 D5}. Soit 21.8\textcelsius, 21.3\textcelsius ~~et 21.3\textcelsius.}

En résumé, on voit qu'il serait très difficile de faire fonctionner l'équipement sans documentation pour connaître : la vitesse du bus RS-486, l'adresse du secondaire, les adresses des registres utilisés et leur contenu et le codage de l'information dans les registres et les unités utilisées. On a donc un degrés d'interopérabilité faible, les deux entités doivent s'accorder sur un grand nombre de paramètres.

\Question{Taux d'Humidité}
{Quel est le taux d'humidité au moment de la mesure ? La documentation précise qu'il s'agit d'un pourcentage avec une précision au dizième de pourcent. }
{On avait lu la valeur \texttt{01 D4} dans le registre 0x02, soit 468 en décimal. D'où un taux d'humidité de 46.8\%}
    \vspace{1em}

On a pu communiquer avec l'objet en utilisant les paramètres par défaut. Mais pour l'insérer dans un bus, il faut pouvoir modifier certains paramètres. La vitesse de transmission et le codage des octets doit être le même, des secondaires ne peuvent pas avoir la même adresses sur le bus.
Le XY-MD02 dispose aussi de \textit{holding register} permettant de le paramétrage comme le montre la table~\vref{tab-XY-HR}


\begin{table}
\begin{center}
\begin{tabular}{|c|c|c|c|}
\hline
 \rowcolor{purple!10} Register Type & Register Address & Register Contents & Bytes \\ \hline \hline
 \multirow{9}{*}{Holding register} & 0x0101 & Device Address & 2 \\ \cline{2-4}
                                 & 0x1202 & Bit rate: & 2 \\
                                 &        & \tabitem 0~: 9600  & \\
                                 &        & \tabitem 1~: 14400 & \\
                                 &        & \tabitem 2~: 19200 & \\  \cline{2-4}
                                 & 0x0103 & Temperature correction & 2 \\ 
                                 &         & -10\textcelsius ~~- 10\textcelsius &  \\ \cline{2-4}
                                 & 0x0104 & Humidity correction & 2 \\ 
                                 &         & -10\%RH - 10\%RH &  \\ \hline
\end{tabular}
\end{center}
\caption{Holding Register d'un XY-MD02}
\label{tab-XY-HR}
\end{table}

\subsection{Passerelle IP}

\begin{wrapfigure}{r}{8cm}
\centerline{\includegraphics[width=.5\columnwidth]{Pictures/encaps3.png}}
\end{wrapfigure}

Il est possible d'étendre la portée d'un réseau Modbus en ajoutant une passerelle IP. Cela correspond au troisième méthode d'interconnexion de la figure~\vref{fig-encap}. La passerelle, connectée sur le bus où restent connectés les secondaires, possède une adresse IP. Le primaire ouvre une connexion TCP avec la passerelle et y envoie ses requêtes. La passerelle recopie les données sur le bus. Inversement, les réponses des objets sont retournées à la passerelle qui les envoie au primaire en utilisation la connexion TCP. La figure~\vref{fig-gwmodbus} illustre les échanges. On note l'ouverture de connexion TCP qui se fait au démarrage du primaire qui reste active pour toute les échanges. On peut aussi remarquer que les messages TCP sont acquittés.

\begin{figure}
    \centering
    
    \begin{tikzpicture}
    
    %\clip (0.0, 0) rectangle (16,10);
    %\draw[help lines] (0,0) grid (15,9); 
    
    \node[inner sep=0pt] (meter) at (1,7)
        {\includegraphics[width=.1\textwidth]{meter.jpg}};
        
    \node[inner sep=0pt] (gw) at (7,7)
        {\includegraphics[width=.2\textwidth]{modbusGW2.png}};
    
    \node[inner sep=0pt] (primary) at (13,7)
        {\includegraphics[width=.2\textwidth]{Pictures/QModMaster.png}};
        
        
    \coordinate (tline) at (0, 5.4);
    
    \draw (meter |- tline) node {\small{Compteur}};
    \draw (gw |- tline) node {\small{Passerelle}};
    \draw (primary |- tline) node {\small{Primaire}};
    
    \coordinate (wline) at (0, 8.3);
    
    \draw [decorate, decoration=snake, blue] (meter |- wline) -- \coordinate(a) (gw |- wline);
%    \draw [decorate, decoration={snake}, mirror, yellow] (gw |- wline) -- coordinate(a) (meter);
    
    \draw (a) node [below] {\small{RS-485}};
    
    \path(gw) -- coordinate(b) (primary);
    
    \draw [very thick] ([xshift=0.2cm]gw |- wline) -- coordinate(c) (primary |- wline);
     \draw (c) node [below] {\small{Internet}};
     
     \coordinate(cline) at (0, 5);
     
     \draw [->, thin] (meter |- cline) -- ++ (0, -4) ;
     \draw [->, thin] (gw |- cline) -- ++ (0, -4);
     \draw [->, thin] (primary |- cline) -- ++ (0, -4); 
     
     \draw [thick, ->] ([yshift=-0.1cm] primary |- cline) coordinate (d) -- ([yshift=-0.3cm] gw |- cline) coordinate (e);
     
     \draw (d) node [right] {\tiny{SYN}};
     \draw (e) node [left] {\tiny{SYN/ACK}};
     \draw [thick, ->] (e) -- ([yshift=-0.5cm] primary |- cline) coordinate (f);
     
    \draw (f) node [right] {\tiny{ACK}};
    \draw [thick, ->] (f) -- ([yshift=-0.7cm] gw |- cline) coordinate (h);
    
    \coordinate (tcp_line) at (14.5, 0);
    
    \draw [decoration=brace, decorate] (d -| tcp_line) -- coordinate(i) (h -| tcp_line);
    
    \draw (i) node [below, rotate=90] {\small{Ouverture}};
    
    \draw [double, double distance=0.2cm,] ([yshift=-1.5cm] primary |- cline) coordinate (d) -- ([yshift=-1.7cm] gw |- cline) coordinate (e);
    
    
    \draw  [->] ([yshift=-1.8cm] gw |- cline) -- ([yshift=-2cm] primary |- cline) coordinate(k);
    \draw (k) node [right] {\tiny{ACK}};


    \draw [purple, -> ] ([yshift=-1.5cm] primary |- cline) coordinate (d) -- ([yshift=-1.7cm] gw |- cline);      
    \draw [purple, -> ] ([yshift=-1.7cm] gw |- cline)  -- ([yshift=-2.1cm] meter |- cline) coordinate (e);  

    \draw (d) node [right,purple] {\small{requête}};

    \draw [blue, ] (e) -- ([yshift=-2.5cm] gw |- cline);    
    \draw [double, double distance=0.2cm ] ([yshift=-2.5cm] gw |- cline) -- ([yshift=-2.7cm] primary |- cline);    
    \draw [blue, -> ] ([yshift=-2.5cm] gw |- cline) -- ([yshift=-2.7cm] primary |- cline) coordinate(m);   
    
    \draw  [->] ([yshift=-2.8cm] primary |- cline) -- ([yshift=-3cm] gw |- cline) coordinate(l);
    \draw (l) node [left] {\tiny{ACK}};
    
    \draw (m) node [right,blue] {\small{réponse}};

   
    \end{tikzpicture}
    
    \caption{Passerelle entre le réseau Internet et Modbus.}
    
    \label{fig-gwmodbus}
\end{figure}

    \vspace{1em}


Comme dans l'exemple du capteur de température, les spécifications du \Index{compteur électrique} sont nécessaires pour comprendre la signification des registres utilisables. Le compteur code ses valeurs sur des nombres flottant sur 32 bits dont le codage est spécifié par la norme \Index{IEEE 754}\footnote{\url{https://en.wikipedia.org/wiki/IEEE_754}}. Ces valeurs sur 32 bits doivent être codées sur deux registres consécutifs d'où l'incrémentation de 2 en 2 que l'on retrouve sur la tableau, le compteur pouvant mesurer trois phase électriques.


\begin{table}
\begin{center}
\begin{tabular}{|c|c|c|c|c|}
\hline
 \rowcolor{purple!10} Register Type & Register Address & Register Contents & Unité & Format \\ \hline \hline
 \multirow{10}{*}{Input register} & 0x0000 & \Index{Voltage} phase A & V &  IEEE 754 \\ \cline{2-5}
                                 & 0x0002 & Voltage phase B & V &  IEEE 754 \\ \cline{2-5}
                                 & 0x0004 & Voltage phase C & V &  IEEE 754 \\ \cline{2-5}
                                 & 0x0008 & \Index{Intensité} phase A & A &  IEEE 754 \\ \cline{2-5}
                                 & 0x000A & Intensité phase B & A &  IEEE 754 \\ \cline{2-5}
                                 & 0x000C & Intensité phase C & A &  IEEE 754 \\ \cline{2-5}
                                 & 0x0010 & \Index{Puissance} Totale & KWh &  IEEE 754 \\ \cline{2-5}
                                 & 0x0012 & Puissance phase A & KWh &  IEEE 754 \\ \cline{2-5}
                                 & 0x0014 & Puissance phase B & KWh &  IEEE 754 \\ \cline{2-5}
                                 & 0x0016 & Puissance phase C & KWh &  IEEE 754 \\ \cline{2-5}
                                 & 0x0036 & \Index{Fréquence} & Hz &  IEEE 754 \\ \hline

\end{tabular}
\end{center}
\caption{quelques \textit{Input Register} du compteur électrique}
\label{tab-XY-IR}
\end{table}

    \vspace{1em}

\Index{Wireshark} peut capturer une requête ayant circulée sur le réseau Ethernet, coté primaire.


\begin{termc}[backgroundcolor=\color{backcolour}, escapechar=#]
#\texttt{\small{0000 \colorbox{purple!50}{98 d8 63 62 29 49 10 65 30 b0 54 bf 08 00}\colorbox{blue!30}{45 00}   ..cb)I.e0.T...E.}}#
#\texttt{\small{0010  \colorbox{blue!30}{00 34 db ef 40 00 80 06 00 00 c0 a8 01 fc c0 a8}   .4..@...........}}#
#\texttt{\small{0020  \colorbox{blue!30}{01 57}\colorbox{red!30}{e2 c9 01 f6 16 90 37 98 5d 57 a0 fa 50 18}   .W......7.]W..P.}}#
#\texttt{\small{0030  \colorbox{red!30}{02 01 84 ca 00 00} \ul{00 0a} \ul{00 00} \ul{00 06} 1c \textcolor{blue}{04} \textcolor{red}{00 00}   ................}}#
#\texttt{\small{0040  \textcolor{green}{00 02}   ..              }}#                                 
\end{termc}

On retrouve les encapsulations des protocoles Ethernet, IP et TCP, suivi des données TCP. Elles se composent de  trois champs qui n'existent pas dans la requête circulant sur le bus RS-485~:
\begin{itemize}
    \item le numéro de la transaction sur deux octets incrémenté à chaque requête,
    \item la version du protocole sur deux octets, 
    \item la longueur en octet de la transation.
\end{itemize}

    \vspace{1em}

Les champs suivants sont identique à ceux de la trame sur le bus RS-485~:
\begin{itemize}
    \item l'adresse du secondaire sur un octet, ici 0x1c ou 28,
    \item la nature de la requête : 0x04 pour lire un input register,
    \item le registre à lire sur deux octets, ici 0x0000 correspondant à la tension sur la phase A.
    \item le nombre de registre à lire, ici 2 pour obtenir les 32 bits de la valeur.
\end{itemize}

    \vspace{1em}

Le primaire reçoit la réponse suivante~:

\begin{termc}[backgroundcolor=\color{backcolour}, escapechar=#]
#\texttt{\small{0000  \colorbox{purple!50}{10 65 30 b0 54 bf 98 d8 63 62 29 49 08 00}\colorbox{blue!30}{45 00}   .e0.T...cb)I..E.}}# 
#\texttt{\small{0010  \colorbox{blue!30}{00 35 c9 75 00 00 40 06 2c aa c0 a8 01 57 c0 a8}   .5.u..@.,....W..}}# 
#\texttt{\small{0020  \colorbox{blue!30}{01 fc}\colorbox{red!30}{01 f6 e2 c9 5d 57 a0 fa 16 90 37 a4 50 18}   ......]W....7.P.}}# 
#\texttt{\small{0030  \colorbox{red!30}{44 70 e1 6e 00 00} \ul{00 0a} \ul{00 00} \ul{00 07} 1c \textcolor{blue}{04} \textcolor{orange}{04}\colorbox{black}{\textcolor{white}{43}}   Dp.n...........C}}# 
#\texttt{\small{0040  \colorbox{black}{\textcolor{white}{69 9e 4a}} \ \ \ \ \ \ \ \ \ \ \ \ \ \ \ \ \ \ \ \ \ \ \ \ \ \ \ \ \ \ \ \ \ \ \ \ \ \ \ i.J}}# 
                           
\end{termc}

On y retrouve, après les encapsulations protocolaires d'Ethernet, IP et TCP les données suivantes~:
\begin{itemize}
    \item le numéro de transaction qui correspond à celui employé dans la requête précédente. cela permet de faire le lien entre les deux messages qui aurait pu se defaire en cas de perte de paquets sur le réseau Internet,
    \item la version du protocole,
    \item la longueur de la réponse, ici 7 octets,
    \item l'adresse du secondaire ayant répondu, ici 28,
    \item la nature de la requête,
    \item le nombre d'octets retournés, ici 4,
    \item et la valeur des deux registres \texttt{0x43699e4a} qui correspond à un nombre flottant tel que le représente le standard IEEE 754. Il existe de nombreux sites sur Internet qui permettent la conversion\footnote{\url{https://www.h-schmidt.net/FloatConverter/IEEE754.html}}. Comme le montre la figure~\vref{fig-ieee754}, on obtient la valeur \SI{233.61831665}\volt qui correspond bien à une tension offerte par un réseau électrique.
\end{itemize}

  \begin{figure}[tbp]
\centerline{\includegraphics[width=1\columnwidth]{Pictures/IEEE754.png}}
\caption{Conversion d'un nombre flottant}
\label{fig-ieee754}
\end{figure}

\Question{Requête Modbus TCP}
{Soit la requête suivante:


Quel est le numéro de port utilisé par Modbus TCP.}
{0x1f6, soit 502 en décimal}

\Question{Requête Modbus TCP}
{Quelle valeur de registre est demandée.}
{la fréquence en Hz}

\Question{Réponse Modbus TCP}
{Soit la réponse suivante


Comment peut-on vérifier que la réponse peut correspondre à la requête précédente.}
{Le numéro de séquence XXXX est le même.}


\Question{Réponse Modbus TCP}
{Quelle valeur est retournée. Est-ce cohérent ?}
{}
%%%%%%%%%%%%%%%%%%%%%%%%%%%%%%%%%%%%%%%%%%%%%%%

% Architecture de l'IoT

%%%%%%%%%%%%%%%%%%%%%%%%%%%%%%%%%%%%%%%%%%%%%%%


\newpage
%\chapter{ARCHITECTURE POUR L'IOT}

\section{Introduction}
  
    \vspace{1em}
   
 \begin{wrapfigure}{r}{3cm}
\Youtube{https://youtu.be/DjRhnbg0FjY}
\end{wrapfigure}

Les objets se caractérisent 
par 
une capacité
de traitement limitée et par une consommation énergétique réduite pour préserver l’autonomie imposée par une alimentation sur batterie.
Or, les activités les plus consommatrices pour un équipement sont l’émission et la réception de données. 
Pour maximiser l’autonomie des équipements, il faut revoir l’intégralité des protocoles, mais en les calquant sur les architectures existantes pour en assurer la compatibilité. 

La figure~\vref{fig-pile-IoT} reprend un certain nombre d'adaptation protocolaires, à différents niveau du modèle \ac{ISO}, capable de s'adapter aux caractéristiques des objets contraints. Dans les chapitres suivants nous reviendrons sur ces technologies en partant de la représentation des données pour aller jusqu'aux couches basses.

\begin{figure}[tbp]
\centerline{\includegraphics[width=1\columnwidth]{Pictures/Capture17.png}}
\caption{Pile protocolaire de l'IoT}
\label{fig-pile-IoT}
\end{figure}

\section{Topologies}

Les réseaux pour l’internet des objets peuvent être divisés en deux catégories : les topologies \Index{maillé}es (\Index{Mesh} in english) et étoilées (\Index{star}).


\subsection*{Réseaux Maillés}

Les réseaux maillés, tels que la famille IEEE 802.15.4,  sont une adaptation d’un protocole d’accès Wi-Fi pour préserver l’énergie. La portée de transmission est limitée à 50 mètres pour limiter la consommation d'énergie ; et par conséquent les messages doivent être relayés par d’autres nœuds pour atteindre leur destination.

Le débit est de quelques centaines de kilobits/s et la taille de la trame est de quelques centaines d’octets. 

Ces réseaux sont performants pour transporter des données IoT, mais le protocole de routage, ainsi que le relayage des trames, consomment l’énergie des objets.

\subsection*{Réseaux en Étoile}

Les topologies en \Index{étoile} ne nécessitent pas de tels mécanismes de routage. Toutes les communications se font avec un point central qui relaie les informations vers la destination.

Les progrès réalisés dans le traitement des signaux permettent d’étendre la portée de transmission à faible puissance. Cette famille de réseaux est appelée réseaux étendus à faible puissance (\ac{LPWAN}) comme \Index{Sigfox}, \Index{LoRaWAN}, ou même du côté de la téléphonie cellulaire avec des évolutions de la norme \Index{4G} et une intégration plus complète dans la \Index{5G}. Le \rfc{8376} donne, en anglais, un aperçu de ces techniques.


    \vspace{1em}

Avec une puissance de transmission de 25 mW, il est possible de communiquer sur une distance de 3 km en milieu urbain et de 20 km dans un environnement dégagé. Les \ac{LPWAN} sont compatibles avec les appareils de classe 0 car ils ne nécessitent pas la mise en place d’une pile \ac{IP}. La figure ci-dessous décrit une architecture typique pour les \ac{LPWAN}.

\begin{figure}[tbp]
\centerline{\includegraphics[width=1\columnwidth]{Pictures/TolologieStar.jpg}}
\caption{Architecture simplifiée des réseaux LPWAN}
\label{fig-topo-star}
\end{figure}


L’appareil envoie des données brutes sur le réseau radio. Le signal radio est capté par une ou plusieurs passerelles radio, et la trame est envoyée à une passerelle réseau (\ac{LNS}  pour les réseaux \Index{LoRaWAN}, et \ac{SCEF} pour les réseaux \ac{3GPP}).

Le propriétaire de l’appareil a associé l’appareil à un connecteur dans le LPWAN \ac{NGW} qui peut être un \ac{URI}, une adresse de broker \ac{MQTT} ou une Web socket. Lorsque l’appareil envoie des données, il est relié à l’application par ce tunnel.

Certaines technologies telles que LoRaWAN ou Sigfox utilisent des bandes sans licence, imposant un cycle d’utilisation (\Index{duty cycle}) de 0,1 à 10 \% selon les canaux pour assurer l’équité entre les nœuds, empêchant ainsi qu'un équipement ne monopolise le canal de transmission. Comme cette restriction s’applique également à l’antenne du fournisseur, la communication entre le réseau et l’appareil est considérablement limitée.

L’utilisation principale de ces réseaux \ac{LPWAN} est la télémétrie où un appareil envoie régulièrement des informations ou une alarme de temps en temps (par exemple des capteurs de température). Le débit et la taille des messages est beaucoup plus réduit que dans le cas de réseaux maillés.

    \vspace{1em}

\section{Niveaux 1 et 2}

 Concernant le niveau 2, le but est de gagner en énergie lors des transmissions. Déjà on peut dire adieu à Ethernet car cela imposerait d'utiliser l'infrastructure filaire et donc on ne pourrait pas placer les objets où on veut, surtout s'ils se déplacent. Les communications par ondes radio sont privilégiées. 
 
 Pour l'Internet des objets, le \Index{Wi-Fi} est également trop gourmand en énergie. On lui préfère donc une évolution appelée \Index{IEEE 802.15.4} qui reprend son principe de fonctionnement mais l'adapte à un faible débit et à des trames de petite taille. En particulier pour économiser l'énergie, la portée est réduite à une dizaine de mètres et il faut généralement utiliser des relais pour atteindre une destination. 
 
 \Index{Bluetooth} a été adapté pour des objets avec une basse consommation \ac{BLE}. 
     \vspace{1em}

 Côté téléphonie cellulaire, les protocoles évoluent pour prendre en compte les objets. La norme 4G a intégrée les communications à plus bas débit. La 5G inclura une classe permettant des communications avec les objets économes en énergie et réduisant les temps de latence. 
 
\section{IP et couches d'adaptation}

  Au niveau 3, on va vers l'utilisation plus massive d'\ac{IPv6} puisque la version 4 a son espace d'adressage saturé. la taille de l’adresse est étendue sur 128 bits offrant $2^{96}$ fois plus d’adresses.
  
  \begin{figure}[tbp]
\centerline{\includegraphics[width=1\columnwidth]{Pictures/Capture18.png}}
\caption{Format d'un en-tête IPv6}
\label{fig-IPv6-header}
\end{figure}

  
  
  
  
  
  Mais, comme on le voir sur la figure~\vref{fig-IPv6-header} \ac{IPv6} implique des en-têtes plus grandes, ce qui est gênant car les réseaux de niveau 2 transportent de plus petites trames.
  Une \Index{couche d'adaptation} entre la couche \ac{IP} et le niveau 2 est nécessaire puisque les niveaux 2 conçus pour l'internet des objets ne peuvent pas transporter naturellement de grands paquets. Deux actions sont mises en œuvre : \Index{compression} de la taille des en-têtes pour réduire leur impact, et \Index{fragmentation} pour découper le paquet en petites trames si la première mesure ne suffit pas. 


     \vspace{1em}

Il existe deux grandes familles de couche d'adaptation :

\begin{itemize}
 \item \Index{6LoWPAN} \rfc{4944}, \rfc{6282}, qui va intégrer un mécanisme de compression de l'en-tête \ac{IPv6} et de fragmentation pour envoyer un gros paquet divisé en petites trames. En effet, dans un réseau maillé, il n'est pas possible de se priver d'informations fournies par la couche \ac{IP} car les nœuds intermédiaires en ont besoin pour acheminer le message vers le destinataire. 
6LoWPAN est sans état et compresse toutes les en-têtes IPv6 sans configuration. 
\item \ac{SCHC} (prononcer chic) \rfc{8724} va imposer des règles décrivant l'en-tête du message et va envoyer le numéro de la règle en remplacement de l'en-tête. La compression est beaucoup plus importante et peut porter sur plusieurs couches protocolaires. Cependant, pour la mettre en œuvre, il faut avoir une idée des flux qui vont circuler sur le réseau. \ac{SCHC} est spécifié pour les réseaux en \Index{étoile} et plus particulièrement les \ac{LPWAN}.

\end{itemize}

\section{Mise en \oe{}uvre de \Index{REST}}
  
  Au-dessus on avait vu que comme \ac{HTTP} était le protocole dominant, \ac{TCP} l'était aussi. Mais pour l'IoT ce n'est pas optimal. En effet TCP/HTTP sont des protocoles complexes qui demandent beaucoup de mémoire. Pour réduire l'impact de la pile protocolaire, l'\ac{IETF} a défini un nouveau protocole appelé \ac{CoAP} qui demande que quelques Kilo Octets pour fonctionner. \ac{CoAP} repose sur \ac{UDP} ce qui simplifie encore la mise en oeuvre. 
  
  
       \vspace{1em}

  Pour poursuivre dans l’intégration des objets dans l’internet, le protocole \ac{CoAP} \rfc{7252} se substitue à \ac{HTTP}. Il en reprend le mécanisme de nommage, d’utilisation des ressources, et les primitives de manipulation entre un client et un serveur.

La capacité de traitement du capteur et son alimentation  en énergie sont souvent très limitées. 
La grande force de \ac{CoAP} est d’être :
\begin{itemize}
\item facile à mettre en œuvre. Les mises en œuvre de \ac{CoAP} nécessitent peu de mémoire ;
\item entièrement compatible avec 
\ac{HTTP} et il est possible d’aller d’un 
protocole à l’autre au travers de  passerelles génériques, c’est-à-dire non liées 
à un usage particulier (comme le montre la figure~\vref{fig-encap}).
\end{itemize}
De ce fait, \ac{CoAP} va manipuler des ressources, identifiées par des \ac{URI}. Il est donc possible d'ancrer les données fournies par les objets dans l'écosystème actuel des communications entre ordinateurs, fortement structuré autour des principes REST.

     \vspace{1em}

La sécurité, en particulier le chiffrement des données, suit aussi les mêmes chemins que l’internet traditionnel. 
Il existe un chiffrement au-dessus d’\ac{UDP} qui, à l’instar
de \ac{HTTPS}, 
chiffre les échanges.
  
\section{Representation des données}

Pour la structuration des données, \ac{XML} n'est pas utilisée car il est trop bavard. \ac{JSON} est beaucoup plus efficace pour transporter des informations structurées. Il existe un équivalent binaire que nous verrons par la suite \ac{CBOR} qui est beaucoup plus performant, et simple à mettre en oeuvre est compatible avec JSON.

\section {Alternatives à REST}


Il n'est pas obligé de tout mettre en oeuvre tous les protocoles définis par l'IETF. Il est possible également d'y intégrer des protocoles spécifiés pour un métier. 

   \begin{wrapfigure}{r}{7cm}
\centerline{\includegraphics[width=.4\columnwidth]{Pictures/linky.png}}
\end{wrapfigure}

Par exemple, le compteur électrique \Index{Linky} que tous les Français connaissent en implémente qu'une partie. Au lieu d'utiliser \ac{CoAP}, les électriciens utilisent leurs propres applications suivant la norme \ac{DLMS}/\ac{Cosem}. Celle-ci repose sur \ac{UDP} puis \ac{IPv6} et \Index{6LoWPAN} et finalement sur une variante de \Index{IEEE 802.15.4} adaptée pour transporter l'information sur les câbles électriques.


\newpage
%\chapter{LA REPRESENTATION DE LA DONNEE}

\section{Introduction}
     \vspace{1em}

Envoyer une donnée sur un réseau n’est pas aussi simple que l’on croit.
     \vspace{1em}

 \begin{wrapfigure}{r}{3cm}
\Youtube{https://youtu.be/k-RhgiwKx2M}
\end{wrapfigure}
Il faut faire la différence entre le format utilisé pour stocker des données dans la mémoire de l'ordinateur et celui employé pour l'envoyer à une autre machine. En effet, chaque machine à sa propre représentation souvent liée aux capacités de leur processeur. Cela est surtout vrai pour les nombres. Ils peuvent être stockés sur un nombre de bits plus ou moins important ou peuvent être représentés en mémoire de manière optimisée pour accélérer leur traitement. 

En revanche, la représentation des chaînes de caractères (non accentués) est relativement uniforme car elle se base sur le code ASCII qui est le même pour tous les ordinateurs. Un texte de base est facilement compréhensible par toutes les machines. Une solution serait donc de n'utiliser que des chaînes de caractères. 

     \vspace{1em}

Par exemple, si l’on veut envoyer l'entier ayant pour valeur 123, il existe plusieurs représentations possibles :
\begin{itemize}
\item envoyer une chaîne de caractères ”123” contenant les chiffres du nombre ;
\item envoyer la valeur binaire 1111011.
\end{itemize}

     \vspace{1em}

On voit que juste pour transmettre une simple valeur stockée dans la mémoire d'un ordinateur, il existe plusieurs options et évidemment pour que cette valeur soit interprétée de la bonne façon, il faut que les deux extrémités se soient mises d'accord sur une représentation.

Quand on veut transmettre plusieurs valeurs, c'est-à-dire quand on a des données structurées, d'autres problèmes surviennent.

Par exemple : quelle est la taille des blocs que l’on va transmettre ? Comment indiquer la fin de la transmission ? Pour une chaîne de caractères, comment indiquer qu’elle se termine ? Autre exemple : si l'on veut transmettre "12" puis "3", comment faire pour que l'autre extrémité ne comprenne pas "123" ?

     \vspace{1em}


Pour que la transmission se fasse correctement, il faut que l’émetteur et le récepteur adoptent les mêmes conventions. Quand il s’agit d’un ensemble de données, il faut être capable de les séparer. Avec les tableurs, une première méthode est possible avec la notation \ac{CSV} Comme son nom l’indique, les valeurs sont séparées par des virgules. Les valeurs sont représentées par des chaînes de caractères. Les textes sont différenciés des valeurs numériques, par l’utilisation de guillemets. Ainsi, 123 sera interprété comme un nombre et ”123” comme un texte.

Si cette représentation est adaptée aux tableurs, elle est relativement pauvre car elle ne permet de représenter que des valeurs sur des lignes et des colonnes. Pour les usages du Web, il a fallu trouver un format plus souple permettant de représenter des structures de données complexes. Évidemment, comme rien n'est simple, il en existe plusieurs et les applications échangeant des données devront utiliser le même.

     \vspace{1em}


On voit que l'envoi de la chaîne de caractères ne suffit pas, il faut la formater pour que le récepteur puisse trouver le type de la donnée transmise, qu'un nombre ne soit pas interprété comme une chaîne de caractères, qu'une chaîne de caractères reste une chaîne de caractères même si elle ne contient que des chiffres.   
    \vspace{1em}
   
\section{La \Index{sérialisation}}

Sous ce nom barbare se cache la méthode utilisée pour transmettre 
des données d’un ordinateur à un autre.
Une donnée peut être simple
(un nombre, un texte) ou plus complexe (un tableau, une structure...). 
Elle est stockée dans la mémoire de l'ordinateur suivant une représentation qui lui est propre. Par exemple, la taille des entiers peut varier d'une technologie de processeur à une autre, l'ordre des octets dans un nombre peut aussi être différente (little et big endian). 
Pour des structures complexes comme les tableaux, les éléments peuvent être rangés à différents emplacements de la mémoire. 

La sérialisation consiste à transformer une structure de données en une séquence qui pourra être transmise sur le réseau, stockée dans un fichier ou une base de données. L'opération inverse, consistant à reconstruire localement une structure de données, s'appelle désérialisation. 

Il existe plusieurs formats pour sérialiser les données. Ils peuvent être binaires mais ceux généralement utilisés sont basés sur des chaînes de caractères. En effet, la représentation \ac{ASCII} définissant les caractères de base et codée sur 7 bits est commune à l'ensemble des ordinateurs. L'autre avantage du code ASCII est qu'il est facilement lisible et simplifie la mise au point des programmes.  

Wikipédia donne ce tableau (cf. figure~\vref{fig-ASCII}) des codes \ac{ASCII} datant de 1972 (une éternité en informatique) et recolorisé par nos soins.

\begin{figure}[tbp]
\centerline{\includegraphics[width=1\columnwidth]{Pictures/Capture20.png}}
\caption{Codage ASCII des caractères}
\label{fig-ASCII}
\end{figure}

Les caractères en orange ne sont pas imprimables. Ils permettent de contrôler la communication des données ou de gérer l'affichage en revenant à la ligne. On les reconnait car la séquence binaire commence par 00X XXXX. On rappelle que le code ASCII est sur 7 bits ; le bit supplémentaire (bit de parité) conduisant à 1 octet était utilisé pour détecter des erreurs de transmission. Les valeurs de 0x30 à 0x39 codent les chiffres de 0 à 9. 

\subsection*{Hexlify}

En Python, il existe le module \Index{binascii} très pratique qui permet de convertir une séquence binaire en une chaîne de caractères ou inversement :
\begin{itemize}
\item \pfunction{binascii}{hexlify} prend un tableau d'octets et le convertit en une chaîne de caractères hexadécimaux plus lisible pour les spécialistes. Cela permet de visualiser n'importe quelle séquence de données. 
\item \pfunction{binascii}{unhexify} fait l'inverse. Il prend une chaîne de caractères et la convertit en un tableau d'octets. Cela peut vous faciliter la programmation car, dans votre code, il est plus facile de manipuler des chaînes de caractères.

\end{itemize}
Dans la suite, nous l'utiliserons pour manipuler des identifiants. Par exemple, ce bout de code illustre l'utilisation de ces fonctions :

\begin{python}
mac = lora.mac()
print ('devEUI: ',  binascii.hexlify(mac))

# create an OTAA authentication parameters
app_eui = binascii.unhexlify('70 B3 D5 7E D0 03 3A E3'.replace(' ',''))

\end{python}

Comme nous le verrons par la suite, a fonction \pfunction{network}{lora.mac()} retourne un tableau d'octets. La fonction \pfunction{binascii}{hexlify} ligne suivante le convertit en chaîne de caractères pour un affichage plus propre. 

Inversement, nous devons affecter une séquence binaire à la variable \texttt{app\_eui}. Nous mettons cette séquence hexadécimale en chaîne de caractères. Les espaces offrent plus de lisibilité. Ils sont retirés par la méthode replace et le résultat est converti en binaire grâce à \pfunction{binascii}{unhexify}

\section{Base64}

Le passage d'une séquence binaire à une chaîne de caractères ASCII en représentant les valeurs conduit à un doublement du volume. Chaque bloc de 4 bits va conduire à produire un octet correspondant au caractère d'un chiffre ou d'une lettre de A à F. Le reste des codes n'est pas utilisé.

Le codage base64 offre un meilleur rendement en utilisant 64 bits pour coder les valeurs. Un dictionnaire fait la correspondance entre 64 valeurs et un caractère ASCII. Cependant, si l'on veut coder 4 octets, soit 32 bits, il faudra 5 blocs de 6 bits, et il y aura deux bits restants. Le symbole = indique que 2 bits sont ajoutés à la fin du codage. Donc, dans notre cas, il faudra ajouter deux symboles = comme le montre la figure ci-dessous :

\begin{figure}[tbp]
\centerline{\includegraphics[width=1\columnwidth]{Pictures/Capture21.png}}
\caption{Codage Base64 de données binaires}
\label{fig-base64}
\end{figure}

On notera que pour les petites séquences, ce codage n'est pas meilleur que la transformation de la séquence hexadécimale en chaîne de caractères. Ici, il faut 8 caractères pour coder 4 octets. 

Il existe beaucoup d'outils en ligne pour faire les conversions entre ces différentes représentations, comme le site \url{www.asciitohex.com}.

\subsection*{Python module: base64}

En Python3, le module base64 permet de faire ces conversions.  Ce module est un peu susceptible sur les types de données à utiliser.

\begin{python}[numbers=left,numbersep=5pt]
import base64

val = b"\x01\x0234"
ser = base64.b64encode(val)
print (ser)
print (ser.decode())
ori = base64.b64decode(ser)
print (ori)
\end{python}

qui donne à l’exécution :

\begin{termc}[backgroundcolor=\color{backcolour}]
b'AQIzNA=='
AQIzNA==
b'\x01\x0234'
\end{termc}

À noter que l'utilisation du \texttt{ser.\pfunction{str}{decode()}}, ligne 6, pour transformer une chaîne d'octets en chaîne de caractères, c'est-à-dire supprimer le \texttt{b} du début, peut être utilisé dans certains cas.



\section{HTML}

La sérialisation en chaînes de caractères (par exemple en Python via la commande \pfunction{binascii}{hexlify}) ou en Base64 concerne surtout des données binaires. Mais la donnée peut être aussi structurée, par exemple la page d'un tableur. Il faut donc formater le document pour éviter une fusion des différents champs.


\ac{HTML}, sans entrer dans les détails, définit un format où les champs sont repérés par un balisage.  Une balise de début est un mot clé entre \texttt{<>} et, pour une \Index{balise} de fin, le mot clé est précédé du caractère \texttt{/}. Par exemple, la figure~\vref{fig-HTML} avec le balisage, le premier paragraphe est formaté de cette manière, Dans le MOOC :

\begin{figure}[tbp]
\centerline{\includegraphics[width=1\columnwidth]{Pictures/Capture22.png}}
\caption{Codage HTML d'une page Web}
\label{fig-HTML}
\end{figure}

Les balises peuvent aussi prendre des arguments, comme la balise \Index{span} dans l'exemple précédent. Ainsi, si l'on regarde une page Web, comme indiqué figure~\vref{fig-Web-HTML}, le navigateur est capable de l'analyser pour trouver les \ac{URI} qu'elle contient. La balise \texttt{\Index{img}} indiquant qu'il s'agit d'une image, le client peut interroger le serveur pour l'afficher à l'écran. Ce format structuré de sérialisation nous permet de mettre en place une caractéristique de \ac{REST}, c'est-à-dire les liens entre ressources.

\begin{figure}[tbp]
\centerline{\includegraphics[width=1\columnwidth]{Pictures/Capture23.png}}
\caption{Capture d'une page Web}
\label{fig-Web-HTML}
\end{figure}


\section{XML}

Si \ac{HTML} est dédié au formatage à l'écran de données textuelles et à la navigation sur le Web. \ac{XML}\footnote{\url{https://www.w3.org/TR/xml/}} défini par le \ac{W3C}, est un format d’échange entre deux applications. Par exemple, pour échanger les notes des étudiants entre la plate-forme FUN et une autorité de certification des cours, on pourrait utiliser le format suivant~:

\begin{termc}[backgroundcolor=\color{palerod}]
<etudiant>
   <prenom>John</prenom>
   <nom>Deuf</nom>
   <note>18</note>
</etudiant>

\end{termc}

Il est facile en lisant l'exemple de trouver le prénom, le nom et la note de l'étudiant. On peut noter qu'il n'y a pas de différence entre la note et le nom de l'élève. Il s'agit de caractères.

     \vspace{1em}

S'il est syntaxiquement correct, rien ne dit que le créateur fournit quelque chose de correct qui pourra être interprété par une autre instance.  \ac{XML} peut inclure une grammaire ou un schéma qui est utilisé pour valider que les informations représentées dans le fichier sont non seulement syntaxiquement conformes au langage \ac{XML}, mais aussi conformes au schéma. Ce schéma va décrire les champs attendus et leur type (texte, nombre...). Vous pouvez accéder à ce cours si vous voulez en savoir plus sur les schémas \ac{XML}.

Du point de vue de l’internet des objets, même si le XML pourrait être un bon candidat pour l’échange d’informations, il est un format trop lourd et donc énergivore. On peut noter que pour envoyer une note sur 20 qui, dans l'absolu, prendrait 6 bits, on transmet \texttt{<note>18</note>}, soit 15 caractères soit 120 bits! 

\section{JSON}

 \begin{wrapfigure}{r}{3cm}
\Youtube{https://youtu.be/IhZ9w6jWnq8}
\end{wrapfigure}

\ac{JSON}  offre un moyen de structurer l’information de manière plus compacte que \ac{XML}. JSON s’impose comme le langage commun pour échanger les informations. A l’origine, JSON était utilisé par\Index{Javascriptt} pour échanger des informations ; par exemple, pour afficher en temps réel l’évolution des cours de la bourse ou pour afficher des graphiques dynamiques sur l’écran de l’utilisateur.

JSON \rfc{8259} est un format d’échange simple. Il définit 4 types de données :

\begin{itemize}
    \item nombre~: Les nombres sont composés de chiffres et peuvent être positifs, négatifs, entiers ou flottants.
    \item texte~: Le texte est délimité par des guillemets simples ou doubles.
    \item \Index{tableau}~: Les tableaux sont des listes d’éléments séparés par des virgules et entourés de crochets.
    \item \Index{objet}~: L’objet est une liste de paires composées d’une \Index{clé} et d’une valeur. La clé est une chaîne de caractères et la valeur peut être de n’importe quel type. La clé doit être unique à l’intérieur d’un objet, et référence entièrement la valeur qui la suit. Le couple clé - valeur est séparé par le caractère 2 points  \texttt{:}. Les éléments de l’objet sont séparés par des virgules. L'objet est délimité par des accolades.
\end{itemize}

Par exemple, quelques structures \ac{JSON} :

\begin{itemize}
    \item \texttt{[1, -2, 0.3, 4e1]} est un tableau qui contient 4 nombres ;
    \item \texttt{[1, ”2”, ”34”]} est un tableau contenant un nombre et deux chaines de caractères ;
    \item \texttt{[1, [2, 3 , ”4”]]} est un tableau de deux éléments dont le second est également un tableau de 3 éléments ;
    \item \texttt{\{ ”couleur” : [34, 16, 3]\}} est un objet qui contient un élément et la valeur est un tableau ; 
    \item \texttt{\{ ”name” : ”bob”, ”age” : 30\}} est un objet qui contient deux éléments référencés par les chaînes de caractères (ou index)  ”name” et ”age”.
    
    L’ordre dans lequel sont placés les éléments est indifférent. \texttt{\{”age” : 30, ”name” : ”bob”\}} est équivalent au dernier exemple. 

    Cela impose que l’index utilisé pour accéder à une valeur doit être unique dans la structure objet \texttt{\{”name” : ”bob”, ”name” : ”alice”}\} est interdit.
\end{itemize}

     \vspace{1em}

Le listing suivant donne un exemple de structure JSON tirée du \rfc{8259}. Il contient un objet JSON avec une seule clé \texttt{”Image”}. La valeur de cette clé est une autre structure qui contient six éléments. 

\begin{termc}[backgroundcolor=\color{palerod}, language=json]
{
"Image": {
      "Width": 800,
      "Height": 600,
      "Title": "View from 15th Floor",
      "Thumbnail": {   
           "Url": "http://www.example.com/image/481989943",
           "Height": 125,
           "Width": 100
       },
       "Animated" : false,
       "IDs": [11, 943, 234, 38793]
    }
}
\end{termc}

Le balisage par clé est un élément fondamental dans la structure des données. Il est primordial d'être cohérent et d'assurer une concordance entre émetteur et récepteur sur l'intitulé de la clé pour pouvoir récupérer l'information voulue. De la même façon, il faut s'accorder sur les unités de mesure : une interprétation d'une mesure en centimètre alors qu'elle est en pixel peut être désastreux ; c'est un problème d'interopérabilité.
     \vspace{1em}

JSON est facilement exploitable dans d’autres langages. Par exemple en Python, le module JSON peut être utilisé pour convertir une structure JSON qui est une chaîne ASCII en une représentation interne Python. Les tableaux sont convertis en listes et les objets en dictionnaires.

     \vspace{1em}
     \pythonlst{example\_json.py}

Le programme \pprog{example\_json.py} reprend la structure précédente. La variable \texttt{struct\_python} est une structure Python. On peut voir que les valeurs pour \texttt{"Animated"} et \texttt{"Copyright"} sont les mots clé Python \texttt{False} (avec un F majuscule) et \texttt{None}. Le programme affiche deux fois cette valeur avec la commande standard \texttt{print} puis avec le module \pfunction{pprint}{pprint} pour avoir un affichage plus lisible. On peut remarquer que l'ordre d'affichage des clés est différent. Comme \texttt{"Title"} était défini deux fois, seul le dernier est conservé dans la structure Python.

\begin{termc}[backgroundcolor=\color{palerod}, language=json, basicstyle=\tiny]
{'Image': {'IDs': [17, 2371, 234, 38793], 'Height': 600, 'Animated': False, 'Title': 'Empty picture', 'Thumbnail': 
{'Url': 'http://www.example.com/image/481989943', 'Width': 100, 'Height': 125}, 'Width': 800, 'Copyright': None}}
{'Image': {'Animated': False,
           'Copyright': None,
           'Height': 600,
           'IDs': [17, 2371, 234, 38793],
           'Thumbnail': {'Height': 125,
                         'Url': 'http://www.example.com/image/481989943',
                         'Width': 100},
           'Title': 'Empty picture',
           'Width': 800}}
\end{termc}

Grâce à la fonction \pfunction{json}{dumps} du module \texttt{json}, la variable \texttt{struct\_python} est transformée en JSON. Les mots clé  \texttt{False} et  \texttt{None} sont remplacés par  \texttt{false} et  \texttt{null}. Le programme affiche une chaîne de caractères.

\begin{termc}[backgroundcolor=\color{palerod}, language=json, basicstyle=\tiny]
{"Image": {"IDs": [17, 2371, 234, 38793], "Height": 600, "Animated": false, "Title": "Empty picture", "Thumbnail": 
{"Url": "http://www.example.com/image/481989943", "Width": 100, "Height": 125}, "Width": 800, "Copyright": null}}
\end{termc}

Pour le retransformer, de JSON en variable Python, on utilise la fonction inverse \pfunction{json}{loads} qui traduit une chaîne de caractères en variable Python.

\begin{termc}[backgroundcolor=\color{palerod}, language=json, basicstyle=\tiny]
{'Image': {'Animated': False,
           'Copyright': None,
           'Height': 600,
           'IDs': [17, 2371, 234, 38793],
           'Thumbnail': {'Height': 125,
                         'Url': 'http://www.example.com/image/481989943',
                         'Width': 100},
           'Title': 'Empty picture',
           'Width': 800}}
\end{termc}


Les autres langues de programmation possèdent également leur propre bibliothèque pour effectuer la traduction.

     \vspace{1em}

Par rapport à \ac{XML}, \ac{JSON} est beaucoup plus permissif et manque de formalisme pour décrire la structure. \ac{JSON-LD} défini par le \ac{W3C} renforce l’interopérabilité de JSON en introduisant des clés spécifiques décrivant la structure des données, une référence aux unités, etc. Nous verrons ces concepts dans la suite du cours.

\section{CBOR}

\begin{wrapfigure}{r}{3cm}
\Youtube{https://youtu.be/thSWuJ-1ld0}
\end{wrapfigure}

\ac{JSON} et \ac{CBOR} sont tous les deux des modes de codage de la donnée.

\ac{JSON} introduit une notation très flexible permettant de représenter toutes les structures de données. Le choix de l'\acs{ASCII} rend ce format universel et n'importe quel ordinateur pourra le comprendre. Mais l'utilisation de l'\acs{ASCII} ne permet pas de transmettre de manière optimale l'information sur un réseau. Quand les réseaux ont un débit raisonnable, cela ne pose pas de problème. Quand on en vient à l'internet des objets, il faut prendre en compte la capacité de traitement limité des équipements et la faible taille des messages échangés.

     \vspace{1em}

Ainsi, en ASCII, la valeur \texttt{123} est codée sur 3 octets (un octet par caractère) tandis qu'en binaire elle n'occuperait qu'un seul octet : \texttt{0111 1011}. 

      \vspace{1em}

\ac{CBOR}, défini dans le \rfc{8949}, permet de représenter les structures de \ac{JSON} mais suivant une représentation binaire. Comme nous le verrons par la suite, si \ac{CBOR} est complètement compatible avec \ac{JSON}, il est possible de représenter d'autres types d'information très utiles dans l'Internet des Objets.

La taille de l'information est réduite et le traitement simplifié. Il faut savoir un peu jongler avec la représentation binaire mais cela reste basique.

      \vspace{1em}

CBOR définit 8 types majeurs qui sont représentés par les 3 premiers bits d'une structure CBOR (cf. figure~\vref{fig-cbor-majeur}). Ces types majeurs ont donc des valeurs comprises entre 0 et 7 (\texttt{000} à \texttt{111} en binaire).

\begin{figure}[tbp]
\centerline{\includegraphics[width=1\columnwidth]{Pictures/cbor1.png}}
\caption{Définition des majeurs en CBOR}
\label{fig-cbor-majeur}
\end{figure}


Les cinq bits suivants contiennent soit une valeur soit une longueur indiquant combien d'octets sont nécessaires pour coder la valeur. \ac{CBOR} offre ainsi des optimisations qui permettent de réduire la longueur totale de la structure des données comme nous le verrons par la suite en étudiant les différents types majeurs.

\subsection{CBOR en python}
Les exemples qui vont suivre peuvent être testés sur votre ordinateur avec Python3. Si une erreur se produit au moment de la définition du module \texttt{\Index{cbor2}}, vous devez l'installer sur votre ordinateur en tapant la commande :

\begin{termc}[backgroundcolor=\color{gray!10}, language=json, basicstyle=\small, escapechar=@]
# @\texttt{pip3 install cbor2}@
\end{termc}


\subsection{Type Entier Positif}
\ac{JSON} ne fait pas de différence entre les nombres, entiers, décimaux, positifs ou négatifs. \ac{CBOR} réintroduit une distinction pour optimiser la représentation.

      \vspace{1em}

Le premier type majeur correspond aux entiers positifs. Il est codé par 3 bits à 0 ; les 5 bits suivants finissent l'octet et, suivant leur valeur, vont avoir une signification différente :
\begin{itemize}
    \item de 0 à 23, il s'agit de la valeur de l'entier à coder ;
    \item 24 indique que l'entier est codé sur 1 octet qui sera codé dans l'octet suivant ;
    \item 25 indique que l'entier est codé sur 2 octets qui seront codés dans les deux octets suivants ;
    \item 26 indique que l'entier est codé sur 4 octets qui seront codés dans les quatre octets suivants ;
    \item 27 indique que l'entier est codé sur 8 octets qui seront codés dans les huit octets suivants.
\end{itemize}

      \vspace{1em}

On peut noter qu'il n'y a pas de surcoût pour coder un entier de 0 à 23. Ainsi, la valeur 15 sera codée 0x0F (\texttt{000-0 1111}) tandis que, pour toutes les autres valeurs supérieures, le surcoût ne sera que d'un octet. La valeur 100 sera codé \texttt{000-1 1000}\footnote{\texttt{11000} correspond à 24} suivi du codage sut 1 octet de la valeur 100 (\texttt{0110 0100}).

      \vspace{1em}

\pythonlst{cbor-integer-ex1.py}

Le programme \texttt{cbor-integer-ex1.py}  affiche les puissances de $10$ entre $10^0$ et $10^{18}$~:

\begin{itemize}
    \item Ligne 1, le programme importe le module \texttt{\Index{cbor2}} et le renomme pour plus de simplicité \texttt{cbor}.
    \ligne 5, la boucle permet d'avoir les multiples de 10 (variable \texttt{v)}. 
    \item ligne 6, le module \texttt{cbor} utilise comme pour JSON la méthode \pfunction{cbor2}{dumps} pour sérialiser une structure interne de Python dans la représentation demandée. À l'inverse, la méthode \pfunction{cbor2}{loads} sera utilisée pour importer une structure CBOR dans une représentation interne.
    \item Ligne 7, le \texttt{print} permet d'aligner les données pour que l'affichage soit plus clair ; entre les accolades, le premier chiffre indique la position dans les arguments de format ; le second, après le \texttt{:}, le nombre de caractères. Par exemple, \textttt{\{1:30\}} indique l'argument \texttt{v} de format affiché sur 30 caractères.
\end{itemize}
 
       \vspace{1em}

Le programme donne le résultat suivant~:

\begin{termc}[backgroundcolor=\color{palerod}, language=json, basicstyle=\small, escapechar=@]
 # @\textbf{python3 cbor-integer-ex1.py}@
  0                              1 01
  1                             10 0a
  2                            100 1864
  3                           1000 1903e8
  4                          10000 192710
  5                         100000 1a000186a0
  6                        1000000 1a000f4240
  7                       10000000 1a00989680
  8                      100000000 1a05f5e100
  9                     1000000000 1a3b9aca00
 10                    10000000000 1b00000002540be400
 11                   100000000000 1b000000174876e800
 12                  1000000000000 1b000000e8d4a51000
 13                 10000000000000 1b000009184e72a000
 14                100000000000000 1b00005af3107a4000
 15               1000000000000000 1b00038d7ea4c68000
 16              10000000000000000 1b002386f26fc10000
 17             100000000000000000 1b016345785d8a0000
 18            1000000000000000000 1b0de0b6b3a7640000
\end{termc}

       \vspace{1em}


On voit facilement que les valeurs 1 et 10 sont codées sur 1 octet ; que 100 est codé sur 2 octets tandis que les valeurs 1 000 et 10 000 sont codées sur 3 octets. Les valeurs entre 100 000 et 1 000 000 000 nécessitent 5 octets et les valeurs suivantes, 9 octets.

       \vspace{1em}


La taille de la représentation s'adapte à la valeur. Ainsi, il n'est pas nécessaire de définir une taille fixe pour coder une donnée.

On peut aussi noter que comme le type majeur est sur 3 bits, ce type peut être reconnu dans une lecture hexadécimale du résultat car la séquence commence toujours par le symbole \texttt{0} ou \textttt{1}..

\subsection{Type Entier Négatif}

Le type majeur entier négatif est à peu près similaire à l'entier positif. Le type majeur est \texttt{001} et le codage de la valeur se fait sur la valeur absolue du nombre à laquelle on retranche 1. Cela évite deux codes différents pour les valeurs 0 et -0.

       \vspace{1em}

Ainsi, pour coder -15, on va coder la valeur 14, ce qui donne en binaire 001-1 1110. Ainsi, -24 peut également être codé sur 1 octet tandis que +24 sera codé sur 2 octets.

\pythonlst{cbor-integer-ex2.py}

Le programme \texttt{cbor-integer-ex2.py} reprend le même code que le programme précédent, mais la variable \texttt{v} est initialisée avec la valeur -1. Ce programme va traiter les puissances de 10 négatives.

\begin{termc}[backgroundcolor=\color{palerod}, language=json, basicstyle=\small, escapechar=@]
  # @\textbf{python3.5 cbor-integer-ex2.py}@
  0                             -1 20
  1                            -10 29
  2                           -100 3863
  3                          -1000 3903e7
  4                         -10000 39270f
  5                        -100000 3a0001869f
  6                       -1000000 3a000f423f
  7                      -10000000 3a0098967f
  8                     -100000000 3a05f5e0ff
  9                    -1000000000 3a3b9ac9ff
 10                   -10000000000 3b00000002540be3ff
 11                  -100000000000 3b000000174876e7ff
 12                 -1000000000000 3b000000e8d4a50fff
 13                -10000000000000 3b000009184e729fff
 14               -100000000000000 3b00005af3107a3fff
 15              -1000000000000000 3b00038d7ea4c67fff
 16             -10000000000000000 3b002386f26fc0ffff
 17            -100000000000000000 3b016345785d89ffff
 18           -1000000000000000000 3b0de0b6b3a763ffff
\end{termc}


\subsection{Type Séquence binaire ou Chaîne de caractères}

Les séquences binaires et les chaînes de caractères ont le même comportement. Le type majeur est respectivement \texttt{010} et \texttt{011}. Il est suivi par la longueur de la séquence ou de la chaîne. Le même type de codage que pour les entiers est utilisé :
\begin{itemize}
    \item si la longueur est inférieure à 23, elle est codée dans la suite du premier octet. On trouve ensuite le nombre d'octets ou de caractères correspondant à cette longueur ;
    \item si la longueur peut être codée dans 1 octet (donc inférieure à 255), la suite du premier octet contient 24 puis l'octet suivant contient la longueur suivie du nombre d'octets ou de caractères correspondant.
    \item si la longueur peut être codée dans 2 octets (donc inférieure à 65535), la suite du premier octet contient 25 puis l'octet suivant contient la longueur suivie du nombre d'octets ou de caractères correspondant.
    \item si la longueur peut être codée dans 4 octets, la suite du premier octet contient 26 puis l'octet suivant contient la longueur suivie du nombre d'octets ou de caractères correspondant.
    \item si la longueur peut être codée dans 8 octets, la suite du premier octet contient 27 puis l'octet suivant contient la longueur suivie du nombre d'octets ou de caractères correspondant.
\end{itemize}

       \vspace{1em}

Ce codage est aussi assez optimal. Il est rare d'envoyer plus de 23 caractères.

\pythonlst{cbor-string.py}

Le programme \texttt{cbor-string.py} montre la représentation de chaînes de caractères de longueur croissante ainsi qu'une séquence binaire~:

\begin{itemize}
    \item ligne 3, la variable \texttt{i} prend des valeurs de 1 à 9.
    \item ligne 6, La multiplication d'une chaîne de caractères par un entier (ligne 4) indique le nombre de répétitions de celle-ci.
    \item lignes 8 et 9 montrent le codage d'une chaîne d'octets. La variable bs contient la représentation en CBOR d'une chaîne d'octets Python (représenté par le caractère \texttt{b} avant les guillemets, les valeurs qui ne correspondent pas à des caractères ASCII sont précédées des symboles \texttt{$\backslash$x}). La représentation en hexadécimal de l'objet CBOR est ensuite affichée.
\end{itemize}

       \vspace{1em}

 
Le résultat est le suivant :

\begin{termc}[backgroundcolor=\color{palerod}, language=json, basicstyle=\tiny, escapechar=@]
# @\textbf{python3.5 cbor-string.py}@
  1 674c6f526157414e
  2 6e4c6f526157414e4c6f526157414e
  3 754c6f526157414e4c6f526157414e4c6f526157414e
  4 781c4c6f526157414e4c6f526157414e4c6f526157414e4c6f526157414e
  5 78234c6f526157414e4c6f526157414e4c6f526157414e4c6f526157414e4c6f526157414e
  6 782a4c6f526157414e4c6f526157414e4c6f526157414e4c6f526157414e4c6f526157414e4c6f526157414e
  7 78314c6f526157414e4c6f526157414e4c6f526157414e4c6f526157414e4c6f526157414e4c6f526157414e4c6f526157414e
  8 78384c6f526157414e4c6f526157414e4c6f526157414e4c6f526157414e4c6f526157414e4c6f526157414e4c6f526157414e4c6f526157414e
  9 783f4c6f526157414e4c6f526157414e4c6f526157414e4c6f526157414e4c6f526157414e4c6f526157414e4c6f526157414e4c6f526157...
43010203
\end{termc}


Jusqu'à 3 répétitions de la chaîne de caractères "LoRaWAN", le codage de la longueur est optimal (codé sur 2 octets).

\subsection{Type tableau}

Le type tableau va regrouper un ensemble d'éléments. Chacun de ces éléments étant une structure CBOR, la seule information nécessaire pour connaître le début et la fin d'un tableau est son nombre d'éléments. Le type majeur est \texttt{100}. Il existe deux méthodes pour coder la longueur d'un tableau :
\begin{itemize}
    \item si celle-ci est connue au moment du codage, il suffit de l'indiquer avec un codage identique à celui utilisé pour indiquer la longueur d'une chaîne de caractères ;
    \item si celle-ci n'est pas connue au moment du codage, il existe un code spécial pour indiquer la fin du tableau. Nous en reparlerons par la suite.
\end{itemize}

\pythonlst{cbor-array.py}

Le programme \texttt{cbor-array.py} donne quelques exemples de codage de tableau :
\begin{itemize}
    \item \texttt{[1,2,3,4]} défini ligne 3, devient \texttt{8401020304}. On peut deviner la structure du message CBOR : \texttt{0x84} indique un tableau de 4 éléments (attention le décodage n'est pas toujours aussi simple). Les 4 éléments sont des entiers inférieurs à 23 ;
    \item \texttt{[1,[2, 3], 4]} défini ligne 7 devient \texttt{8301\ul{820203}04}. Il s'agit d'un tableau de 3 éléments dont le deuxième est un tableau de deux éléments ;
    \item \texttt{[1000, +20, -10, +100, -30, -50, 12]} défini ligne 11, devient \texttt{871903e814291864 381d38310c}. On peut noter que le codage des éléments est de longueur variable, mais comme chaque élément code sa longueur, il est juste nécessaire d'en connaître le nombre.
\end{itemize}

\subsection{Type map (Liste de paires)}



Le type Liste de paires ou Map est indiqué par la valeur \texttt{101}. Il fonctionne de la même manière que les tableaux en comptant le nombre d'éléments. Mais cette fois-ci, la valeur représente une paire, c'est-à-dire deux objets CBOR consécutifs.

\pythonlst{cbor-mapped.py}

Le programme \texttt{cbor-mapped.py} donne un exemple d'encodage. A noter que la structure à encoder n'est pas directement compatible avec JSON\footnote{json.\pfunction{json}{dumps aurait converti les clés numériques en chaînes de caractères \texttt{'\{"type": "hamster", "2": "program", "taille": 300, "15": 113\}'}}}, certaines clés ne sont pas des chaînes de caractères.

       \vspace{1em}


Le résultat est \texttt{a464747970656768616d73746572667461696c6c6519012c026770726f6772 616d0f1871} ce qui n'est pas très facile à lire. 


\subsubsection{cbor.me}

\begin{wrapfigure}{r}{3cm}
\Youtube{https://youtu.be/h1XnaFy_FoI}
\end{wrapfigure}

Le site web \url{https://cbor.me} permet de faire automatiquement le codage dans un sens ou dans l'autre.
La colonne de gauche représente la donnée en JSON et celle de droite en CBOR (dite "représentation canonique" qui facilite la lecture). En ayant entré la séquence hexadécimale ci-dessus, le site la présente comme indiqué figure~\vref{fig-cbor-me}.
La partie CBOR est indexée et commentée pour rendre l'objet CBOR plus lisible. Il peut également être traduit dans un équivalent JSON, bien que certaines clés restent numériques.

\begin{figure}[tbp]
\centerline{\includegraphics[width=1\columnwidth]{Pictures/cbor-me.png}}
\caption{Définition des majeurs en CBOR}
\label{fig-cbor-me}
\end{figure}

Sur ces exemples, on peut voir que CBOR est beaucoup plus permissif et complet que JSON, le premier champ des map CBOR peut être numérique et n'a pas à être unique dans toute la structure. Néanmoins  CBOR
définit un mode strict dans lequel ces clés doivent être codées en ASCII et unique pour être compatibles avec
JSON. Si une clé est répété plusieurs fois dans une structure CBOR, il traiter directement l'information dans la structure CBOR et ne pas chercher à la convertir ou la désérialiser car il y a un risque de perte d'information. 

\subsection{Type étiquette}

CBOR enrichit le typage des données ; ce qui permet de manipuler plus facilement des données. Par exemple, une chaîne de caractères peut représenter une date, une URI, voire une URI codée en base 64. Le type \texttt{110} peut être suivi d'une valeur ou \Index{tag} dont une liste exhaustive est maintenue parl'IANA\footnote{\url{https://www.iana.org/assignments/cbor-tags/cbor-tags.xhtml}}.

\pythonlst{cbor-tag.py}

Par exemple, le programme \texttt{cbor-tag.py} retourne les résultats suivants:

\begin{termc}[backgroundcolor=\color{palerod}, language=json, basicstyle=\small, escapechar=@]
# @\textbf{python3.5 cbor-tag.py}@
2018-05-22
c074323031382d30352d32325430303a30303a30305a
2018-05-22 00:00:00+00:00
43010203
<class 'datetime.datetime'>
\end{termc}


La représentation canonique montre plus facilement le tag dans la séquence binaire :
\begin{termc}[backgroundcolor=\color{palerod}, language=json, basicstyle=\small, escapechar=@]
C0                                      # tag(0)
   74                                   # text(20)
      323031382D30352D32325430303A30303A30305A # "2018-05-22T00:00:00Z"
\end{termc}

Le tag 0 implique un format normalisé pour la date ; d'où l'ajout des heures, minutes et secondes, alors qu'elles n'ont pas été spécifiées initialement. On peut également remarquer que \pfunction{cbor2}{loads} retourne un type \texttt{date} et non une chaîne de caractères.

\subsection{Le type flottant et valeurs particulières}
Le dernier type majeur (111) permet de coder les nombres flottants en utilisant la représentation définie par l'\Index{IEEE 754}. Suivant la taille de la représentation, la suite de l'octet contient les valeurs 25 (demi précision sur 16 bits), 26 (simple précision sur 32 bits) ou 27 (double précision sur 64 bits).

Ce type permet également de coder les valeurs définies par JSON : True (valeur 20), False (valeur 21) ou None (valeur 22).

Finalement, ce type peut indiquer la fin d'un tableau ou d'une liste de paires quand la taille n'est pas connue au début du codage.

\section{Questions sur CBOR}
\Question{Avantages de CBOR}{
Quel sont les avantage de CBOR par rapport à JSON (2 réponses) ?
\begin{itemize}[label=$\square$]
   \item \Correct{Il est plus compact dans la représentation des données.}
   \item \Wrong{Il permet de représenter des nombres flottants.}
   \item \Wrong{Il compresse les chaînes de caractères. }
   \item \Correct{Il est plus simple à implémenter.}
 \end{itemize}
   
 }
{
CBOR ne compresse pas les chaînes de caractères. Il ajoute juste leur longueur. CBOR et JSON codent tous les deux des nombres flottants, ce n'est donc pas un avantage de CBOR. Par contre, le fait d'utiliser des valeurs binaires au lieu de l'ASCII pour représenter les nombres, de ne pas avoir de crochets ou accolades, permet d'avoir une représentation beaucoup plus compacte. Les petites valeurs numériques sont représentées sans surcoût. Réaliser un codeur ou un decodeur de CBOR est beaucoup plus simple qu'en JSON car la représentation des données est beaucoup plus stricte (pas d'espace, pas de retour à la ligne... ). Les deux représentations permettent d'utiliser des nombres flottant donc aucune n'a d'avantage sur ce point.
}

\Question{Flottant}{
Un flottant est-il toujours plus compact en CBOR qu'en JSON ? - Vous pouvez vous aider de \url{https://cbor.me}

\begin{itemize}[label=$\circ$]
   \item \Wrong{Oui, c'est le but de CBOR.}
   \item \Wrong{Oui pour les flottants qui ont la partie décimale à 0..}
   \item \Wrong{Oui pour les flottants de petite précision (jusqu'au centième).}
   \item \Correct{Oui pour les flottants de grande précision (6 chiffres après la virgule).}
 \end{itemize}
}{
Un nombre flottant, quelle que soit sa valeur, est représenté par 8 octets en CBOR. En JSON, un nombre flottant est représenté par une chaîne de caractères. Donc "3.0" nécessite 3 caractères, donc plus compacte que CBOR. Mais "3.1415926" est codé sur 9 caractères donc moins compacte que CBOR.
}

\Question{Chaîne de caractères}{
Soit une chaîne de caractères en CBOR.
\begin{itemize}[label=$\circ$]
   \item \Wrong{Elle est compressée avec un algorithme entropique (e.g. codage de Huffman).}
   \item \Correct{Elle peut contenir des caractères accentués.}
   \item \Wrong{Chaque caractère est codé sur 6 bits.}
 \end{itemize}

}{
Une chaîne de caractères est une séquence d'octets. Il est possible d'utiliser les représentations offrant des caractères accentués (voire des émoticones). Par contre, CBOR ne compresse pas les données.
}

\Question{Taille variable}{
-En CBOR, la taille d'un entier varie en fonction de sa valeur !
\begin{itemize}[label=$\circ$]
   \item \Correct{Vrai}
   \item \Wrong{Faux}
   \item \Wrong{Ça dépend de la manière dont on a déclaré cet entier.}
 \end{itemize}

}{
Oui, un entier inférieur à 23 sera codé sur un seul octet. Pour les valeurs plus grandes, il faut ajouter la longueur.
}

\Question{Tableau}{
En CBOR, un tableau peut contenir des objets de types différents.
\begin{itemize}[label=$\circ$]
   \item \Correct{Vrai}
   \item \Wrong{Faux}
   \item \Wrong{Ça dépend de la manière dont on a déclaré ce tableau.}
 \end{itemize}
}{
Vrai, on a la même flexibilité qu'en JSON en imbriquant n'importe quel type de données dans un tableau.
}

\Question{Fraction}{
On veut définir un tableau de deux éléments comme une fraction. Quel tag devra précéder la structure ?
(vous pouvez vous aider du \rfc{8949}.
}{
Il s'agit du tag 4, voir le chapitre \textbf{3.4.4. Decimal Fractions and Bigfloats} du \rfc{8949}.
}

\section{SenML}

\ac{SenML} est une une spécification qui exploite JSON ou CBOR. Elle liste un ensemble de noms/unités/mesures et les standardise en un nom de clé unique. En utilisant cette standardisation, on facilite l'interopérabilité. Les clés et valeurs sont donc règlementées et typées pour éviter tous conflits d'interopérabilité. Le format est défini dans la \rfc{8428} et repose sur une structure de tableau regroupant des objets comme le montre la figure suivante tirée de la RFC.

\begin{termc}[backgroundcolor=\color{palerod}, language=json, basicstyle=\small, escapechar=@]
[
 {"bn" : "urn:dev:ow:10e2073a01080063", "bt":1.320067464e+09,
  "bu" : "\%RH", "v":21.2},
 {"t":10, "v":21.3},
 {"t":20, "v":21.4},
 {"t":30, "v":21.4},
...
\end{termc}

SenML définit les clés utilisées dans l’objet. Pour avoir une notation compacte, elles sont limitées à 1 ou 2 caractères. Parmi elles, ”bn” indique un nom de base et ”n” le nom d’un appareil. Si plusieurs appareils envoient la partie commune de l’identifiant de l’appareil, on peut mettre le ”bn” pour éviter de le répéter à chaque fois.

Le temps de base (ou ”bt”) est également un moyen de compacter la notation du temps. Le temps (”t”) donne le décalage et conduit à une valeur plus petite comme on le voit dans l’exemple.

L’unité de base (”bu”) indique l’unité par défaut si les autres objets ne portent pas de mot clé indiquant l’unité (”u”).

La \rfc{8428} définit une liste d’unités telles que le kilogramme (”kg”), le volt (”V”), etc. Dans l’exemple, ”%RH” désigne un pourcentage d’humidité relative. Une valeur numérique utilise la lettre ”v”, une chaîne de caractères utilise la touche ”vs”.

CBOR utilise la même structure mais les petits nombres entiers positifs et négatifs sont substitués dans les clés des objets de CBOR : ”bn”, ”bt”, ”bu” seront respectivement représentés par -1, -2 et -3 et ”n”, ”t”, ”u” par +0, +2 et +6.









%%%%%%%%%%%%%%%%%%%%%%%%%%%%%%%%%%%%%



\immediate\closeout\tempfile
\setboolean{Response}{false}
\chapter{Réponses aux questions}
\input{questions}

\printindex
\end{document}